
%footnotes
%\renewcommand{\thefootnote}{\fnsymbol{footnote}}
\renewcommand{\thefootnote}{\fnsymbol{footnote}}

% Maths
%Vectors
\newcommand{\va}{\mathbf{a}} % arbitrary vectors
\newcommand{\vb}{\mathbf{b}}
\newcommand{\vc}{\mathbf{c}}
\newcommand{\vd}{\mathbf{d}}
\newcommand{\ve}{\mathbf{e}}
\newcommand{\vf}{\mathbf{f}}
\newcommand{\vg}{\mathbf{g}}
\newcommand{\vh}{\mathbf{h}}
\newcommand{\vi}{\mathbf{i}}
\newcommand{\vj}{\mathbf{j}}
\newcommand{\vk}{\mathbf{k}}
\newcommand{\vl}{\mathbf{l}}
\newcommand{\vm}{\mathbf{m}}
\newcommand{\vn}{\mathbf{n}}
\newcommand{\vo}{\mathbf{o}}
\newcommand{\vp}{\mathbf{p}}
\newcommand{\vq}{\mathbf{q}}
\newcommand{\vr}{\mathbf{r}}
\newcommand{\vs}{\mathbf{s}}
\newcommand{\vt}{\mathbf{t}}
\newcommand{\vu}{\mathbf{u}}
% \newcommand{\vv}{\mathbf{v}}
\newcommand{\vw}{\mathbf{w}}
\newcommand{\vx}{\mathbf{x}}
\newcommand{\vy}{\mathbf{y}}
\newcommand{\vz}{\mathbf{z}}
\newcommand{\0}{\mathbf{0}}  % The zero vector
\newcommand{\1}{\mathbf{1}}  % The vector of ones

%Matrices
\newcommand{\mA}{\mathbf{A}} % arbitrary vectors
\newcommand{\mB}{\mathbf{B}}
\newcommand{\mC}{\mathbf{C}}
\newcommand{\mD}{\mathbf{D}}
\newcommand{\mE}{\mathbf{E}}
\newcommand{\mF}{\mathbf{F}}
\newcommand{\mG}{\mathbf{G}}
\newcommand{\mH}{\mathbf{H}}
\newcommand{\mI}{\mathbf{I}} %identity matrix
\newcommand{\mJ}{\mathbf{J}}
\newcommand{\mK}{\mathbf{K}}
\newcommand{\mL}{\mathbf{L}}
\newcommand{\mM}{\mathbf{M}}
\newcommand{\mN}{\mathbf{N}}
\newcommand{\mO}{\mathbf{O}}
\newcommand{\mP}{\mathbf{P}}
\newcommand{\mQ}{\mathbf{Q}}
\newcommand{\mR}{\mathbf{R}}
\newcommand{\mS}{\mathbf{S}}
\newcommand{\mSigma}{\mathbf{\Sigma}}
\newcommand{\mT}{\mathbf{T}}
\newcommand{\mU}{\mathbf{U}}
\newcommand{\mV}{\mathbf{V}}
\newcommand{\mW}{\mathbf{W}}
\newcommand{\mX}{\mathbf{X}}
\newcommand{\mY}{\mathbf{Y}}
\newcommand{\mZ}{\mathbf{Z}}

%functions over vectors and matrices
\newcommand{\proj}{\pi}
\newcommand{\transpose}[1]{#1^{\intercal}} %transpose
\newcommand{\inprod}[2]{\langle #1, #2 \rangle} %inner product
\newcommand{\prodscal}[2]{\left\langle#1, #2\right\rangle} %inner product (Vincent)
\newcommand{\Tr}{\mathrm{Tr}} %trace
\newcommand{\TrOf}[1]{\Tr\!\left(#1\right)}
\newcommand{\No}{\mathrm{No}} %norm

%numerical sets
\newcommand{\Z}{\mathbb{Z}} %integers
\newcommand{\N}{\mathbb{N}} %natural numbers
\newcommand{\Q}{\mathbb{Q}} %rational numbers
\newcommand{\R}{\mathbb{R}} %real numbers
\newcommand{\C}{\mathbb{C}} %complex numbers
\newcommand{\F}{\mathbb{F}} %finite field
\newcommand{\K}{\mathcal{K}} %arbitrary set
\renewcommand{\S}{\mathcal{S}} %arbitrary set

%complex numbers
\newcommand{\conj}[1]{\overline{#1}}

%special constants
\newcommand{\iu}{{i\mkern1mu}} %imaginary i

%ideals
\newcommand{\ideal}[1]{\left\langle #1 \right\rangle}
\newcommand{\idealI}{\mathcal{I}}
\newcommand{\idealJ}{\mathcal{J}}
\newcommand{\idealp}{\mathfrak{p}}

\newcommand{\idealNormOf}[1]{N\!\left(#1\right)}

%sets
\newcommand{\cardinOf}[1]{\left\lvert#1\right\rvert}

%arithmetic functions
\newcommand{\mult}{{\mathrm{mult}}}
\newcommand{\mul}{{\mathrm{mul}}}
\newcommand{\add}{{\mathrm{add}}}
\newcommand{\erf}{{\mathrm{erf}}}
\newcommand{\erfc}{{\mathrm{erfc}}}
\newcommand{\inv}[1]{{{#1}^{-1}}}

%size
\newcommand{\size}{{\mathrm{size}}}

%probability
\newcommand{\Prob}{{\mathrm{Prob}}} % probability

%parameters of any generalized encryption scheme
\newcommand{\ctxtSpace}{\mathcal{C}} %ciphertext space
\newcommand{\ptxtSpace}{\mathcal{P}} %plaintext space
\newcommand{\msgSpace}{\mathcal{M}} %message space
\newcommand{\secpar}{\lambda} %security parameter

%elements of any encryption scheme
\newcommand{\sk}{{\mathtt{sk}}}
\newcommand{\pk}{{\mathtt{pk}}}
\newcommand{\rlk}{{\mathtt{rlk}}}
\newcommand{\ct}{{\mathtt{ct}}}
\newcommand{\pt}{{\mathtt{pt}}}

%basic encryption operations
\newcommand{\KeyGen}{{\mathtt{KeyGen}}}
\newcommand{\SKGen}{{\mathtt{SecretKeyGen}}}
\newcommand{\PKGen}{{\mathtt{PublicKeyGen}}}
\newcommand{\EVGen}{{\mathtt{EvaluateKeyGen}}}
\newcommand{\RKGen}{{\mathtt{RelinKeyGen}}}
\newcommand{\BKGen}{{\mathtt{BootKeyGen}}}
\newcommand{\Encrypt}{{\mathtt{Encrypt}}}
\newcommand{\Enc}{\mathtt{Enc}}
\newcommand{\Decrypt}{{\mathtt{Decrypt}}}
\newcommand{\Dec}{\mathtt{Dec}}
\newcommand{\Encode}{{\mathtt{Encode}}}
\newcommand{\Decode}{{\mathtt{Decode}}}
\newcommand{\Pack}{{\mathtt{Pack}}}
\newcommand{\Unpack}{{\mathtt{Unpack}}}

%homomorphic encryption operations
\newcommand{\Add}{{\mathtt{Add}}}
\newcommand{\AddPlain}{{\mathtt{AddPlain}}}
\newcommand{\Sub}{{\mathtt{Sub}}}
\newcommand{\SubPlain}{{\mathtt{SubPlain}}}
\newcommand{\BasicMul}{{\mathtt{BasicMul}}}
\newcommand{\Mul}{{\mathtt{Mul}}}
\newcommand{\MulPlain}{{\mathtt{MulPlain}}}
\newcommand{\ModSwitch}{{\mathtt{ModSwitch}}}
\newcommand{\Relin}{{\mathtt{Relin}}}
\newcommand{\Rescale}{{\mathtt{Rescale}}}
\newcommand{\Rotate}{{\mathtt{Rot}}}
\newcommand{\Shift}{{\mathtt{Shift}}}
\newcommand{\Frob}{{\mathtt{Frob}}}
\newcommand{\Eval}{{\mathtt{Evaluate}}}
\newcommand{\Select}{{\mathtt{Select}}}
\newcommand{\Replicate}{{\mathtt{Replicate}}}
\newcommand{\IsNonZero}{{\mathtt{IsNonZero}}}
\newcommand{\Power}{{\mathtt{IsNonZero}}}

%lattices
\newcommand{\lat}{\mathcal{L}}
\newcommand{\mindist}{{\lambda_1}}
\newcommand{\mindistOf}[1]{{\lambda_1}\!\left(#1\right)}
\newcommand{\sucmin}[1]{{\lambda_#1}}
\newcommand{\sucminOf}[2]{{\lambda_#1}\!\left(#2\right)}
\newcommand{\smoothpar}[1]{{\eta_#1}}
\newcommand{\smoothparOf}[2]{{\eta_#1}\!\left(#2\right)}
\newcommand{\duallat}[1]{{#1^*}}

%number theory
\newcommand{\numfield}{\mathcal{K}} %the number field 
\newcommand{\numfieldDim}{n} %the dimension of the number field over the field of rational numbers
\newcommand{\defpoly}{f} %the definining polynomial of the number field
\newcommand{\intring}{\mathcal{R}} %the ring of integers of a number field
\newcommand{\expfactor}{\delta_{\mathcal{R}}} %expansion factor 
\newcommand{\dual}[1]{\ensuremath{#1^\vee}} %the dual of an ideal
\newcommand{\codif}{\dual{\intring}}
\newcommand{\canemb}{\sigma} %canonical embedding
\newcommand{\canembMat}{\boldsymbol{\Sigma}} %linear operator of the canonical embedding
\newcommand{\invCanembMat}{\inv{\canembMat}} %invers of the linear operator of the canonical embedding
\newcommand{\canembImg}{H} %image of the canonical embedding in the complex vector space
\newcommand{\canembImgBasis}{\mB} %basis of the can. embedding image over real numbers
\newcommand{\realembNum}{{s_1}} %number of real embeddings
\newcommand{\cmplxembNum}{{s_2}} %number of non-cojugate complex embeddings
\newcommand{\cyclpoly}[1]{{\Phi_{#1}}} %cyclotomic polynomial
\newcommand{\cyclpolyOrd}{m} %order of a cyclotomic polynomial
\newcommand{\CRT}{\mathsf{CRT}} %Chinese remainder theorem ring isomorphism
\newcommand{\disc}{\mathrm{disc}}
\newcommand{\ord}{{\mathrm{ord}}} %order of an element
\newcommand{\eulerphi}[1]{\phi\left(#1\right)} %Euler totient function
\DeclareMathOperator{\Gal}{Gal} %Galois group

%lattice problems
\acrodef{BDD}{Bounded Distance Decoding}
\acrodef{SIS}{Short Integer Solution}
\acrodef{SVP}{Shortest Vector Problem}
\acrodef{uSVP}{Unique Shortest Vector Problem}
\acrodef{GapSVP}{Decision Shortest Vector Problem}
\acrodef{SIVP}{Shortest Independent Vectors Problem}
\acrodef{DGS}{Discrete Gaussian Sampling}
\acrodef{LWE}{Learning with Errors}
\acrodef{SLWE}{Search Learning with Errors}
\acrodef{DLWE}{Decision Learning with Errors}
\acrodef{S(I)VP}{Shortest (Independent) Vector Problem}

%lattice acronyms
\acrodef{ANF}{Algebraic Normal Form}
\acrodef{HNF}{Hermite Normal Form}

%LWE parameters
\newcommand{\LWEdim}{n}
\newcommand{\LWEmod}{q}

%RLWE parameters
\newcommand{\RLWE}{\textsf{RLWE}}
\newcommand{\SRLWE}{\textsf{S-RLWE}}
\newcommand{\DRLWE}{\textsf{D-RLWE}}
\newcommand{\RLWEbasicRing}{\intring}
\newcommand{\RLWEdim}{n}
\newcommand{\RLWEmod}{\LWEmod}
\newcommand{\RLWEring}{{\RLWEbasicRing_\RLWEmod}}
\newcommand{\SCGLWE}{\text{SCG}\textsf{-LWE}}

%HE names
\acrodef{HE}{Homomorphic Encryption}
\acrodef{FHE}{Fully Homomorphic Encryption}
\acrodef{SHE}{Somewhat Homomorphic Encryption}
\acrodef{LHE}{Leveled Homomorphic Encryption}
\acrodef{BGV}{Brakerski-Gentry-Vaikuntanathan}
\acrodef{FV}{Fan-Vercauteren}
\acrodef{GSW}{Gentry-Sahai-Waters}
\acrodef{HAO}{Hiromasa-Abe-Okamoto}
\acrodef{LTV}{Lopez-Tromer-Vaikuntanathan}
\acrodef{YASHE}{Yet Another Somewhat Homomorphic Encryption}
%\newcommand{\BCIV}{\mathtt{BCIV}}
%\newcommand{\HEAAN}{\mathtt{HEAAN}}
%\newcommand{\TFHE}{\mathtt{TFHE}}

%FHE/SHE names
\newcommand{\BGV}{\mathtt{BGV}}
\newcommand{\HEAAN}{\mathtt{HEAAN}}
\newcommand{\FV}{\mathtt{FV}}
\newcommand{\BCIV}{\mathtt{BCIV}}
\newcommand{\TFHE}{\mathtt{TFHE}}

%FV parameters
\newcommand{\FVringDim}{n}
\newcommand{\FVctxtMod}{\LWEmod}
\newcommand{\FVptxtMod}{t}
\newcommand{\FVDelta}{\Delta}
\newcommand{\FVrelinBase}{w}
\newcommand{\FVrlkNum}{\ell}
\newcommand{\FVctxtSpace}{\RLWEbasicRing_\FVctxtMod^2}
\newcommand{\FVptxtSpace}{\RLWEbasicRing_\FVptxtMod}
\newcommand{\slots}{\ell}

%BGV parameters
\newcommand{\BGVptxtMod}{t}

%HEAAN parameters
\newcommand{\HEAANscale}{\Delta}

%RNS operations
\newcommand{\FastBconv}{{\mathtt{FastBconv}}}

%message
\newcommand{\msg}{\mathsf{msg}}

%norms
\newcommand{\abs}[1]{\left\lvert #1 \right\rvert}
\newcommand{\norm}[1]{\left\lVert #1 \right\rVert}
\newcommand{\canorm}[1]{\left\lVert #1 \right\rVert^{\textup{can}}}
\newcommand{\infnorm}[1]{\left\lvert#1\right\rvert_\infty}

% various functions
\newcommand{\wt}{\texttt{wt}}

%modular operation
\newcommand{\modq}[1]{\left[ #1 \right]_\RLWEmod} %modulo (R)LWE modulus in the symmetric interval
\newcommand{\modx}[2]{\left[ #2 \right]_{#1}} %modulo any modulus in the symmetric interval
\newcommand{\modp}[1]{\left[ #1 \right]_p} %modulo p in the symmetric interval
\newcommand{\modt}[1]{\left[ #1 \right]_\FVptxtMod} %modulo the FV plaintext modulus in the symmetric interval
\newcommand{\rem}[2]{\left| #2 \right|_{#1}} %remainder after division

%rounding operations
\newcommand{\round}[1]{\left\lfloor #1 \right\rceil} %rounding
\newcommand{\floor}[1]{\left\lfloor #1 \right\rfloor} %floor
\newcommand{\ceil}[1]{\left\lceil #1 \right\rceil} %ceiling

%distributions
\newcommand{\rand}{\xleftarrow{\$}} %sampled from (Vincent)
\newcommand{\from}{\leftarrow} %sampled from
\newcommand{\ufrom}{\xleftarrow{\$}} %sampled uniformly random from
\newcommand{\udist}{\mathcal{U}} %uniform distribution
\newcommand{\normdist}{\Gamma} %normal distribution
\newcommand{\dgaussdist}{\mathcal{DG}} %discrete Gaussian distribution
\newcommand{\keydist}{\chi_{\text{k}}} %key distribution
\newcommand{\errdist}{\chi_{\text{e}}} %error distribution
\newcommand{\stdev}{\sigma} %standard deviation

%indistinguishability of distributions
\newcommand{\cindist}{\stackrel{\rm c}{\approx}} %computational
\newcommand{\sindist}{\stackrel{\rm s}{\approx}} %statistical

%proof start/end
\newcommand{\bproof}{\noindent {\scshape Proof: }}
\newcommand{\eproof}{\ \mbox{} \hfill $\square$\mbox{}\newline}

%circuits
\newcommand{\circuit}{\mathcal{C}}

%boolean operators
\newcommand{\XOR}{\mathtt{XOR}}
\newcommand{\NAND}{\mathtt{NAND}}
\newcommand{\AND}{\mathtt{AND}}
\newcommand{\EQ}{\mathtt{EQ}}
\newcommand{\LT}{\mathtt{LT}}
\newcommand{\OR}{\mathtt{OR}}
\newcommand{\NOT}{\mathtt{NOT}}
\newcommand{\MOD}{\mathtt{MOD}}

%complexity classes
\newcommand{\poly}{\mathrm{poly}}

%technical commands
\newcommand{\remove}[1]{}
\newcommand{\alert}[1]{\textcolor{red}{#1}}

%comments
\newcommand{\que}[1]{{\color{red} {\textbf{question:} #1}\xspace}}
\newcommand{\todo}[1]{\textcolor{red}{TODO: #1}}

%cryptographic acronyms
\acrodef{GC}{Garbled Circuit}
\acrodef{DSPR}{Decisional Small Polynomial Ratio}

%algorithmic acronyms
\acrodef{SVM}{Support Vector Machine}
\acrodef{LLL}{Lenstra Lenstra Lov{\'a}sz}
\acrodef{GMDH}{Group Method of Data Handling}

%math acronyms
\acrodef{CRT}{Chinese Remainder Theorem}
\acrodef{RNS}{Residue Number System}
\acrodef{NTT}{Number Theoretic Transform}
\acrodef{DFT}{Discrete Fourier Transform}

%computer science acronyms
\acrodef{SIMD}{Single Instruction Multiple Data}
\acrodef{SP}{Streaming Processor}
\acrodef{LSB}{Least Significant Bit}
\acrodef{CPU}{Central Processing Unit}
\acrodef{GPU}{Graphics Processor Unit}
\acrodef{SMP}{Symmetric Multiprocessing}
\acrodef{HPC}{High Performance Computing}
\acrodef{APU}{Accelerated Processing Unit}
\acrodef{FPGA}{Field-Programmable Gate Array}



\newcommand\blankfootnote[1]{%
  \begingroup
  \renewcommand\thefootnote{}\footnote{#1}%
  \addtocounter{footnote}{-1}%
  \endgroup
}
\newcommand*{\skipnumber}[2][1]{%
{\renewcommand*{\alglinenumber}[1]{}\State #2}%
\addtocounter{ALG@line}{-#1}}

%%% Local Variables:
%%% mode: latex
%%% TeX-master: "main"
%%% End:
