\subsubsection{Notations}
Vectors will be written in column form and denoted by boldface lower-case letters. The vector containing only $1$'s in its coordinates is denoted by $\1$. We write $\0$ for the zero vector. The set of integers $\{\ell,\dots,k\}$ is denoted by $[\ell,k]$.

For a positive integer $t$, let $\texttt{wt}(t)$ be the Hamming weight of its binary expansion. We denote the set of residue classes modulo $p$ by $\Z_p$ and the class representatives of $\Z_p$ are taken from the half-open interval $[-p/2, p/2)$.

\subsection{Comparison of integers with finite fields operations}
Let $\S$ be a totally ordered set with a binary relation $<$.
For any $x,y \in \S$, we can define the less-than and the equality functions as follows.
\begin{align*}
  \LT_\S(x,y) = 
  \begin{cases}
    1, & \text{if } x < y; \\
    0, & \text{if } x \ge y, \\
  \end{cases}
  \qquad
  \EQ_\S(x,y) = 
  \begin{cases}
    1, & \text{if } x = y; \\
    0, & \text{if } x \neq y. \\
  \end{cases}
\end{align*}
In our case, $\S$ can be thought as a complete set of representatives of a finite field $\F_{p^d}$ of prime characteristic $p\geq 2$.

\subsubsection{Functions over finite fields}\label{subsec:funcff}
The map defined by $\princhar: x \mapsto x^{p^d-1}$ from $\F_{p^d}$ to the binary set $\{0,1\}$ is called the \emph{principal character}. According to Euler's theorem extended to finite fields, it returns $1$ if $x$ is non-zero and $0$ otherwise. It can thus be used to compute the equality function:
\begin{align*}
  \EQ_{\F_{p^d}}(x,y) = 1 - \princhar (x-y).
\end{align*}
Moreover, note that since:
\begin{align}\label{eq:exp_frob}
  a^{p^d-1} = a^{(p-1)(p^{d-1} + \dots + 1)} = \prod_{i=0}^{d-1} (a^{p-1})^{p^i},
\end{align}
$\chi$ can be computed with Frobenius maps ($x \mapsto x^p$) and multiplications.

Every function from $\F_{p^d}^l$ to $\F_{p^d}$ can be interpolated by a unique polynomial which can be defined with Lagrange interpolation and the help of the principal character. 
\begin{lemma}\label{lem:interpolation}
  Every function $f: \F_{p^d}^l \rightarrow \F_{p^d}$ is a polynomial function represented by a unique polynomial $P_f(X_1,\dots,X_{l})$ of degree at most $p^d - 1$ in each variable.
  In particular,
  \begin{align*}
    P_f(X_1,\dots,X_{l}) = \sum_{a_1,\dots,a_l \in \F_{p^d}} f(a_1,\dots,a_l) \prod_{i=1}^{l} \left(1 - \princhar(X_i - a_i)\right)\,.
  \end{align*}
\end{lemma}

\subsubsection{Comparison of integers}
Let $\F_{p^d} = \F_p[X]/\ideal{f(X)}$ for some irreducible monic polynomial $f(X)$ of degree $d$.
Let $d'\leq d$ and $\S \subseteq [0,p^{d'}-1)$, we can map $S$ into $\F_{p^d}$ by using the decomposition of integers in base $p$:

$$\begin{array}{cccc}
    \iota : & \S & \rightarrow & \F_{p^d} \\
            & \displaystyle\sum_{i=0}^{d'-1}a_ip^i & \mapsto & \displaystyle\sum_{i=0}^{d'-1}a_i X^{i-1} 
  \end{array}$$
Note that the map $\iota$ is injective and gives a one-to-one correspondance between $\S$ and $\F_{p^{d}}$ when $d'=d$ and $\S = [0,p^d)$. This is why from now on we will identify integers belonging to $\S$ with their image by $\iota$ and we will thus omit $\iota$ when the situation will be clear from the context.

Let $a,b\in\S$ two integers we would like to compare, once encoded as elements of the field, $a = \sum_{i=0}^{d'-1} a_i X^i$ and $b = \sum_{i=0}^{d'-1} b_i X^i$ with $a_i, b_i \in \F_p$, they lie in the vector subspace $\F_{p}^{d'}$ of $\F_{p^d}$.

The order of the set $\S$ defines a polynomial function in $\F_{p^d}$, which can thus be interpolated by Lemma~\ref{lem:interpolation}.
  However, the degree of this polynomial is at most $p^d-1$, which means that with this method the degree of the polynomial to evaluate is independent of $d'$ and exponential in $d$.

  Tan et al.~\cite{TLWRK20} proposed an alternative approach that extracts coefficients $a_i, b_i \in \F_p$, which is possible due to the following lemma (see \cite[Theorem 2.24]{LN86} for the proof).
  \begin{lemma}\label{lem:extraction}
    The linear transformations from $\F_{p^d}$ to $\F_p$ are exactly the mappings $L_{\alpha}(x) = \Tr_{\F_{p^d}/\F_p}(\alpha x)$ for some $\alpha \in \F_{p^d}$ and all $x \in \F_{p^d}$.
    Furthermore, $L_{\alpha} \ne L_{\beta}$ if $\alpha \ne \beta$.
  \end{lemma}
  In particular, this lemma implies that for any $i \in [0,d'-1]$ there exist $\alpha_i \in \F_{p^d}$ such that $L_{\alpha_i}(a) = a_i$ for any $a \in \F_p^{d'}$.
  This implies that we can construct the following system of equations over $\F_{p^d}$ 
  \begin{align*}
    \begin{pmatrix}
      1       & 1       & \dots   & 1             \\
      X       & X^p     & \dots   & X^{p^{d-1}}   \\
      \vdots  & \vdots  & \ddots  & \vdots        \\
      X^{d-1}     & X^{(d-1)p}    & \dots   & X^{(d-1)p^{d-1}}  \\
    \end{pmatrix}
    \begin{pmatrix}
      \alpha_0      & \alpha_1      & \dots   & \alpha_{d-1}      \\
      \alpha_0^{p}  & \alpha_1^{p}  & \dots   & \alpha_{d-1}^{p}  \\
      \vdots        & \vdots        & \ddots  & \vdots            \\
      \alpha_0^{p^{d-1}}  & \alpha_1^{p^{d-1}}  & \dots   & \alpha_{d-1}^{p^{d-1}}  \\
    \end{pmatrix}
    =
    \mX \mA
    =
    \mI_{d}
  \end{align*}
  where $\mI_d \in \F_{p^d}^{d \times d}$ is the identity matrix.
  Hence, the $i$th column of $\inv{\mX}$ contain all the powers $\alpha_i, \alpha_i^p, \dots, \alpha_i^{p^{d-1}}$, which define the linear map $L_{\alpha_i}$.

  Therefore, instead of comparing $a$ and $b\in\S$ directly as elements of $\mathbb{F}_{p^d}$, we can compare their vector of coefficients $\bm{a} = (a_0,a_1,\ldots,a_{d'-1})$ and $\bm{b} = (b_0,b_1,\ldots,b_{d'-1}) \in \F_p^{d'}$ using the lexicographical order $\bm{<}$ on $\F_{p}^{d'}$ defined by:
   \begin{align*}
      \bm{a} \bm{<} \bm{b} \Leftrightarrow \exists i\in[0,d'-1] \text{ such that } a_i < b_i \text{ and } \forall j > i ~~ a_j = b_j\,,
    \end{align*}
    which results in the following circuit
    \begin{align*}
      \LT_{\F_p^{d'}}(\bm{a}, \bm{b}) = \sum_{i=0}^{d'-1} \LT_{\F_{p}}(a_{i}, b_{i}) \prod_{j=i+1}^{d'-1} \EQ_{\F_{p}}(a_{j}, b_{j}).
    \end{align*}
  This function requires $d'$ evaluations of $\LT_{\F_{p}}$ which has degree at most $p-1$, $d'-1$ evaluations of $\EQ_{\F_{p}}$ which has also degree $p-1$ and $2d'-1$ additional multiplications.
  Thus, if we ignore the complexity of digit extraction, the total complexity of comparing elements of $\F_{p}^{d'}$ in terms of multiplications is $O(d'(p-1))$.

\subsubsection{Comparison of large integers}
When the size of the integers to compare exceeds the size of the field $p^d$, we can reapply the same trick by decomposing them in base $p^{d'}$ and then compare their vector of digits using the lexicographical order $\bm{<}$ on $\left(\F_{p}^{d'}\right)^k$, for some $d' \le d$.

In this way, we represent an integer $a = a_0 + a_1p^{d'} + a_2p^{2d'}+\ldots a_{k-1}p^{(k-1)d'}$ by the vector $\va=(\bm{a}_0,\bm{a}_1,\ldots,\bm{a}_{k-1}) \in \left(\F_{p}^{d'}\right)^k$ of its digits of length $k$. The comparison of two integers $a$ and $b$ is thus equivalent to the comparison of their vector of digits $\va=(\bm{a}_0,\bm{a}_1,\ldots,\bm{a}_{k-1})$ and $\vb=(\bm{b}_0,\bm{b}_1,\ldots,\bm{b}_{k-1})$ using the lexicographical order $\bm{<}$ on $\left(\F_{p}^{d'}\right)^k$ defined as follows
    \begin{align*}
      \va \bm{<} \vb \Leftrightarrow \exists i\in[0,k-1] \text{ such that } \bm{a}_i < \bm{b}_i \text{ and } \forall j > i ~~ \bm{a}_j = \bm{b}_j\,.
    \end{align*}
  As done in~\cite{TLWRK20}, we can employ $\EQ_{\F_{p}^{d'}}$ and $\LT_{\F_{p}^{d'}}$ to compute $\LT(\va, \vb)$ which evaluates the lexicographical order of $\va$ and $\vb$ as follows:
  \begin{align*}
    \LT(\va, \vb) = \sum_{i=0}^{k-1} \LT_{\F_{p}^{d'}}(\bm{a}_{i}, \bm{b}_{i}) \prod_{j=i+1}^{k-1} \EQ_{\F_{p}^{d'}}(\bm{a}_{j}, \bm{b}_{j}).
  \end{align*}
  Notice that the multiplicative depth of this function depends essentially of the products of the equality functions.
  In fact, these products compare subvectors $\va_i = (\bm{a}_i, \bm{a}_{i+1},\dots,\bm{a}_{k-1})$ and $\vb_i = (\bm{b}_i, \bm{b}_{i+1},\dots \bm{b}_{k-1})$ for $i \in [1,k-1]$.
  Thus, we can rewrite $\LT$ as:
  \begin{align}\label{eq:general_lex_order}
    \LT(\va, \vb) = \sum_{i=0}^{k-1} \LT_{\F_{p}^{d'}}(\bm{a}_{i},\bm{b}_{i}) \EQ_{\F_{p}^{d'}}(\va_{i+1}, \vb_{i+1}) 
  \end{align}
  with the equality function $\EQ_{\F_{p}^{d'}}(\va_{i+1}, \vb_{i+1})$ that returns $1$ if $\va_{i+1} = \vb_{i+1}$ and $0$ otherwise.
  % As shown in~\todo{cite our work}, this function can be realized by a constant-depth circuit in the following way.
  % \begin{align}\label{eq:rand_eq_circuit}
  %   \EQ_{\F_\fieldcard, e}(\vx_{i+1},\vy_{i+1}) = 1 - \princhar_{\fieldcard^e}\left(\sum_{j=i+1}^{\ell-1} r_j (x_j - y_j)\right)
  % \end{align}
  % where $r_j$ are uniformly random elements from $\F_{\fieldcard^e}$.
  % This circuit is false-biased with error probability $\fieldcard^{-e}$.
  % We can compute all $\EQ_{\F_\fieldcard, e}(\vx_{i+1},\vy_{i+1})$ using the same number of multiplications as for the single equality using Algorithm~\ref{alg:vector_equalities_circuit}.\todo{What is the complexity?}


%If $\S$ is a complete set of representatives of a finite field $\F_{p^d}$, then the equality function can be easily realized using the principal character of $\F_{p^d}$, namely $\EQ_\S(x,y) = 1 - \princhar_{p^d}(x-y)$.
% Since this function is independent of the choice of $\S$, we can write
% \begin{align*}
%   \EQ_{\F_{p^d}}(x,y) = 1 - \princhar_\fieldcard(x-y).
% \end{align*}
% Unfortunately, the comparison function is not as simple and we have to rely on Lagrange interpolation (Lemma \ref{lem:interpolation}) to compute it. Yet, there are two different ways to evaluate it: either interpolate $\LT_\S$ as a bivariate function over $\S^2$  -- i.e. compute $\LT_\S(x,y)$ -- or transform it as a univariate function in $z=x-y$ -- i.e. compute $\LT_\S(x-y,0)$.

% While the second approach might seems simpler, it requires nonetheless to be able to distinguish between ``postive'' and ``negative'' values. This means that if $\S = [0,p']$ for some $p'\geq 1$, since the values of $z = x-y$ can range within the interval $[-p',p']$, it requires to use a plaintext modulus $p\geq 2p'+1$ while being only able to use ``half'' of it.  


\subsection{Homomorphic Encryption}
We are interested in leveled HE schemes that support SIMD operations on their plaintexts. We refer to such schemes as \emph{SIMD-schemes} and this section aims at giving the necessary background regarding these schemes.

\subsubsection{Cyclotomic fields and Chinese Remainder Theorem}\label{subsec:crt}

Let $m$ be a positive integer and $n = \varphi(m)$ where $\varphi$ is the Euler's totient function. 
Let $\mathcal{K} = \Q(\zeta_{m})$ be the cyclotomic number field constructed by adjoining a primitive $m$-th root of unity $\zeta_{m}\in\C$ to the field of rational numbers. 
The ring of integers of $\mathcal{K}$, denoted by $\intring$, is isomorphic to $\Z[X]/\ideal{\Phi_m(X)}$ where $\Phi_m(X)$ is the $m$-th cyclotomic polynomial. Let $p>1$ be a prime number coprime to $m$, then $\Phi_m(X)$ splits modulo $p$ into $\ell$ irreducible factors of same degree $d$: $\Phi_m(X) = F_1(X)\cdots F_\ell(X) \bmod p$. The degree $d$ is actually the order of $p$ modulo $m$, and $\ell = n/d$. As noticed in \cite{SV14}, the \ac{CRT} states that in this case the following ring isomorphism holds:

\begin{align}\label{eq:crt}
  \intring_p = \Z_p[X]/(\Phi_m(X)) \cong \Z_p[X]/(F_1(X)) \times \ldots \times \Z_p[X]/(F_{\ell}(X))
\end{align}

For each $i \in [1,\ell]$ the quotient ring $\Z_p[X]/(F_i(X))$ is isomorphic to the finite field $\F_{p^d}$. Hence, the isomorphism in~(\ref{eq:crt}) can be rewritten as:
\begin{align*}
  \intring_p \cong \F_{p^d}^\ell.
\end{align*}
We call every copy of $\F_{p^d}$ in the above isomorphism a \emph{slot}. Therefore, every element of $\intring_p$ contains $\ell$ slots, which implies that an array of $\ell$ independent $\F_{p^d}$-elements can be encoded as a unique element of $\intring_p$.
We enumerate the slots according to the enumeration of the polynomials $F_i(X)$'s. Namely, the slot isomorphic to $\Z_p[X]/(F_i(X))$ is referred to as the \emph{$i$th} slot.

Additions and multiplications of $\intring_p$-elements results in the corresponding coefficient-wise operations of their respective slots. In other words, each ring operation on $\intring_p$ is applied to every slot in parallel, this resembles the Single-Instruction Multiple-Data (SIMD) instructions used in parallel computing.

Using multiplication, we can easily define a projection map $\proj_i$ on $\intring_p$ that sends $a \in \intring_p$ encoding slots $(m_0, \dots, m_{\ell-1})$ to $\pi_i(a)$ encoding $(0, \dots, m_i, \dots, 0)$.
In particular, $\proj_i(a) = a g_i$, where $g_i \in \intring_t$ encodes $(0 \dots, 1, \dots, 0)$.
We can generalize this projection for any $I \subseteq \{1,\dots,\ell\}$ to $\proj_I(a) = a g_I$ with $g_I \in \intring_p$ encoding $1$ in the SIMD slots indexed by $I$.\newline

The field $\numfield = \Q(\zeta_{m})$ is a Galois extension and its Galois group $\mathcal{G} = \Gal{(\numfield/\Q)}$ is isomorphic to $\Z_m^\times$  through: $i \mapsto (\sigma_i: X \mapsto X^i)$ where $i \in \Z_m^\times$. The automorphism $\sigma_p$ corresponding to $p$ is called \emph{the Frobenius automorphism} and generates the Galois group $\Gal{(\F_{p^d}/\F_p)}$ of each slot. This means that $\mathcal{F} = <\sigma_p>\subset \mathcal{G}$ partitions the roots of $\Phi_m$  into $\ell$ sets $X_i$ of $d$ elements, each set corresponding to the roots of a factor $F_i$. Therefore the group $\mathcal{H} = \mathcal{G}/\mathcal{F}$ acts transitively on a set of representative $\bar{X}_i$ of each $X_i$ and thus maps a root $\bar{X}_i$ of $F_i$ to a root $\bar{X}_j$ of $F_j$ for $i\neq j$. In other words the elements of $\mathcal{H}$ permute the SIMD slots. However, the order of $\mathcal{H}$ is $n/d = \ell$, which is less than $\ell!$, the number of all possible permutations of the $\ell$ SIMD slots. Nonetheless, it was shown in~\cite{GHS12} that every permutation of SIMD slots can be realized by combination of automorphisms from $\mathcal{H}$, projection maps and additions.\newline

In this work we will focus on SIMD schemes allowing to perform \emph{exact} computations over encrypted datas. Such schemes include $\BGV$~\cite{BGV12} and $\FV$~\cite{FV12} and have the ring $\intring_p$, for $p\geq 2$, as plaintext space.

\subsubsection{Basic setup}

Let $\lambda$ be the security level of an HE scheme.
Let $L$ be the maximal multiplicative depth of homomorphic circuits we want to evaluate.
Let $d$ be the order of the plaintext modulus $p$ modulo the order $m$ of $\intring$.
Assume that the plaintext space $\intring_p$ has $\slots$ SIMD slots, i.e. $\intring_p \cong \F^\ell_{p^d}$.
% For a vector $\va \in \F^k_{p^d}$, we denote the plaintext encoding of $\va$ by $\pt(\va)$. 
The basic part of a SIMD-scheme consists of key generation, encryption and decryption algorithms.

$\KeyGen(1^\lambda, 1^L) \rightarrow (\sk, \pk)$. Given $\lambda$ and $L$, this function generates the secret key $\sk$ and the public key $\pk$.
Note that the public key contains special key-switching keys that help to transform ciphertexts encrypted under other secret keys back to ciphertexts encrypted under $\sk$. 

$\Encrypt(\pt \in \intring_t, \pk) \rightarrow \ct$. The encryption algorithm takes a plaintext $\pt$ and the public key $\pk$ and outputs a ciphertext $\ct$.

$\Decrypt(\ct, \sk) \rightarrow \pt$. The decryption algorithm takes a ciphertext $\ct$ and the secret key $\sk$ and returns a plaintext $\pt$.
For freshly encrypted ciphertexts, the decryption correctness means that $\Decrypt(\Encrypt(\pt, \pk), \sk) = \pt$. 


\subsubsection{Arithmetic operations}

Homomorphic arithmetic operations are addition and multiplication.

$\Add(\ct_1, \ct_2) \rightarrow \ct$. The addition algorithm takes two input ciphertexts $\ct_1$ and $\ct_2$ encrypting plaintexts $\pt_1$ and $\pt_2$ respectively.
It outputs a ciphertext $\ct$ that encrypts the sum of these plaintexts in the ring $\intring_p$.
%It implies that homomorphic addition sums respective SIMD slots of $\pt_1$ and $\pt_2$.

$\AddPlain(\ct_1, \pt_2) \rightarrow \ct$. This algorithm takes a ciphertext $\ct_1$ encrypting a plaintext $\pt_1$ and a plaintext $\pt_2$.
It outputs a ciphertext $\ct$ that encrypts $\pt_1 + \pt_2$.
%As for the $\Add$ algorithm, $\AddPlain$ sums respective SIMD slots of $\pt_1$ and $\pt_2$.

$\Mul(\ct_1, \ct_2) \rightarrow \ct$. Given two input ciphertext $\ct_1$ and $\ct_2$ encrypting plaintext $\pt_1$ and $\pt_2$ respectively, the multiplication algorithm outputs a ciphertext $\ct$ that encrypts the plaintext product $\pt_1 \cdot \pt_2$.
%As a result, homomorphic multiplication multiplies respective SIMD slots of $\pt_1$ and $\pt_2$.

$\MulPlain(\ct_1, \pt_2) \rightarrow \ct$. Given a ciphertext $\ct_1$ encrypting plaintext $\pt_1$ and a plaintext $\pt_2$, this algorithm outputs a ciphertext $\ct$ that encrypts the plaintext product $\pt_1 \cdot \pt_2$.
%As a result, homomorphic multiplication multiplies respective SIMD slots of $\pt_1$ and $\pt_2$.

Using the above operations as building blocks, one can design homomorphic subtraction algorithms.

$\Sub(\ct_1, \ct_2) = \Add(\ct_1, \MulPlain(\ct_2, \pt(-\1))) \rightarrow \ct$. The subtraction algorithm returns a ciphertext $\ct$ that encrypts the difference of two plaintext messages $\pt_1 - \pt_2$ encrypted by $\ct_1$ and $\ct_2$, respectively.

$\SubPlain(\ct_1, \pt_2) = \AddPlain(\ct_1, \pt_2 \cdot \pt(-\1)) \rightarrow \ct$. This algorithm returns a ciphertext $\ct$ that encrypts $\pt_1 - \pt_2$ where $\pt_1$ is encrypted by $\ct_1$.
We can also change the order of arguments such that $\SubPlain(\pt_1, \ct_2)$ returns a ciphertext $\ct$ encrypting $\pt_1 - \pt_2$. 

As shown in Section~\ref{subsec:crt}, the projection map $\proj_I$ can select the SIMD slots indexed by a set $I \subseteq \{0,\dots,\slots-1\}$ and set the rest to zero.
This functionality is homomorphically realized by the $\Select$ function.

$\Select(\ct, I) = \MulPlain(\ct, \pt(\1_I)) \rightarrow \ct'$ where $\1_I$ is a vector having $1$'s in the coordinates indexed by a set $I$ and zeros everywhere else.
Given a ciphertext $\ct$ encrypting SIMD slots $\vm = (m_0,m_1,\dots,m_{\slots-1})$ and a set $I$, this function returns a ciphertext $\ct'$ that encrypts $\vm' = (m'_0,\dots,m'_{\slots-1})$ such that $m'_i = m_i$ if $i \in I$ and $m'_i = 0$ otherwise.

\subsubsection{Special operations}\label{subsec:special_operations}

One can also homomorphically permute the SIMD slots of a given ciphertext and act on them with the Frobenius automorphism. 

$\Rotate(\ct, i) \rightarrow \ct'$ with $i \in [0, \slots-1]$. Given a ciphertext $\ct$ encrypting SIMD slots 
$$\vm = (m_0,m_1,\dots,m_{\slots-1}),$$
the rotation algorithm returns a ciphertext $\ct'$ that encrypts the cyclic shift of $\vm$ by $i$ positions, namely $(m_i,m_{(i+1) \mod \slots},\dots,m_{(i-1) \mod \slots})$.

$\Frob(\ct, i) \rightarrow \ct'$ with $i\in[0,d-1]$. Given a ciphertext $\ct$ encrypting SIMD slots $\vm$ as above, the Frobenius algorithm returns a ciphertext $\ct'$ that encrypts a Frobenius map action on $\vm$, namely $(m^{p^i}_0,m^{p^i}_1,\dots,m^{p^i}_{\slots-1})$.

As discussed in Section~\ref{subsec:funcff}, the $\Frob$ and $\Mul$ operations can be combined to compute the principal character $\princhar$, which turns non-zero values of SIMD slots into $1$ and leaves slots with zeros unchanged.

$\IsNonZero(\ct) \rightarrow \ct'$. Given a ciphertext $\ct$ encrypting SIMD slots $\vm = (m_0,m_1,\dots,m_{\slots-1})$, this function returns a ciphertext $\ct'$ that encrypts:
\[
  (\princhar(m_0), \princhar(m_1), \dots, \princhar(m_{\slots-1})).
\]
% Kim et al.~\cite{TDSC:KLLW16} showed that one can employ the Frobenius map to decrease the multiplicative depth of $\IsNonZero$.
Recall that $\princhar(m) = m^{p^d-1} = \prod_{i=0}^{d-1} (m^{p-1})^{p^i}$ as shown in~(\ref{eq:exp_frob}).
The multiplicative depth of $x^{p-1}$ is equal to $\ceil{\log_2 (p-1)}$.
The multiplicative depth of $x^{p^i}$ is zero as it can be done by the $\Frob$ operation.
In total, $d-1$ $\Frob$ operations are needed to compute $\princhar(m)$.
As a result, the total multiplicative depth of $\IsNonZero$ is
\begin{align}\label{eq:nonzero_depth}
  \ceil{\log_2 (p-1)} + \ceil{\log_2 d}.
\end{align}
Using general exponentiation by squaring, $x^{p-1}$ requires $\floor{\log_2 (p-1)} + \wt(p-1) - 1$ field multiplications.
Since $d-1$ field multiplications are needed to compute $\prod_{i=0}^{d-1} (x^{p-1})^{p^i}$, the total number of multiplications to compute $\princhar(m)$ is:
\begin{align}\label{eq:nonzero_width}
  \floor{\log_2 (p-1)} + \wt(p-1) + d - 2.
\end{align} 

\subsubsection{Cost of homomorphic operations}\label{subsec:cost}
Note that every homomorphic ciphertext contains a special component called \emph{noise} that is removed during decryption.
However, the decryption function can deal only with noise of small enough magnitude; otherwise, this function fails.
This noise bound is defined by encryption parameters in a way that larger parameters result in a larger bound.
The ciphertext noise increases after every homomorphic operation and, therefore, approaches its maximal possible bound.
It implies that to reduce encryption parameters one needs to avoid homomorphic operations that significantly increase the noise.
Therefore, while designing homomorphic circuits, we need to take into account not only the running time of homomorphic operations but also their effect on the noise.  

Table~\ref{table:he_operations} summarizes the running time and the noise cost of the aforementioned homomorphic operations.
Similar to~\cite{C:HalSho14}, we divide the operations into expensive, moderate and cheap.
The expensive operations dominate the cost of a homomorphic circuit.
The moderate operations are less important, but if there are many of them in a circuit, their total cost can dominate the total cost.
The cheap operations are the least important and can be omitted in the cost analysis.

It is worth to note that there are two multiplication functions $\Mul$ (ciphertext-ciphertext multiplication) and $\MulPlain$ (ciphertext-plaintext multiplication).
Since $\Mul$ is much more expensive than $\MulPlain$, the multiplicative depth of a homomorphic circuit is usually calculated with relation to the number of $\Mul$'s.

\begin{table}[t!]
  \centering
  \begin{tabular*}{.5\textwidth}{@{\extracolsep{\fill} } c c c }
    \toprule
    Operation	& Time			& Noise \\
    \cmidrule(lr){1-1}\cmidrule(lr){2-3}
    $\Add$		& cheap			& cheap 	\\
    $\AddPlain$	& cheap			& cheap \\
    $\Mul$		& expensive		& expensive 	\\
    $\MulPlain$	& cheap			& moderate 	\\
    $\Sub$		& cheap			& cheap  \\
    $\SubPlain$	& cheap			& cheap  \\
    $\Select$	& cheap			& moderate \\
    $\Rotate$ 	 & expensive	& moderate \\
    $\Frob$		 & expensive	& cheap \\
    $\IsNonZero$ & expensive    & expensive \\
    \bottomrule
  \end{tabular*}
  \caption{The cost of homomorphic operations with relation to running time and noise growth.}
  \label{table:he_operations}
\end{table}


% Let $\F_\fieldcard$ be a finite field of characteristic $\fieldchar$.
% We define the multiplicative map $\princhar_\fieldcard: \F_\fieldcard \rightarrow \{0,1\}, x \mapsto x^{\fieldcard-1} \mod \fieldchar$, which is called the \emph{principal character}.
% Due to Euler's theorem, $\princhar_\fieldcard(0) = 0$ and $\princhar_\fieldcard(x) = 1$ for any $x \in \F_\fieldcard^\times$.
% Since
% \begin{align}\label{eq:exp_frob}
% a^{t^d-1} = a^{(t-1)(t^{d-1} + \dots + 1)} = \prod_{i=0}^{d-1} (a^{t-1})^{t^i},
% \end{align}
% the principal character can be realized by Frobenius maps and multiplications.

%%% Local Variables:
%%% mode: latex
%%% TeX-master: "main"
%%% End:
