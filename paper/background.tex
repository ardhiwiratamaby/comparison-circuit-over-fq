\subsection{Notation}
	The set of integers $\{\ell,\dots,k\}$ is denoted by $[\ell,k]$.

\subsection{Finite fields}
	Let $\F_\fieldcard$ be a finite field of characteristic $\fieldchar$.
	We define the multiplicative map $\princhar: \F_\fieldcard \rightarrow \{0,1\}, x \mapsto x^{\fieldcard-1} \mod \fieldchar$, which is called the \emph{principal character}.
	Due to Euler's theorem, $\princhar(0) = 0$ and $\princhar(x) = 1$ for any $x \in \F_\fieldcard^\times$.
	%Since
	%\begin{align}\label{eq:exp_frob}
	%	a^{t^d-1} = a^{(t-1)(t^{d-1} + \dots + 1)} = \prod_{i=0}^{d-1} (a^{t-1})^{t^i},
	%\end{align}
	%the principal character can be realized by Frobenius maps and multiplications.
	It is a standard fact in the theory of finite fields that every function from $\F_\fieldcard^\ell$ to $\F_\fieldcard$ can be represented by a unique polynomial, which is defined with the help of the principal character. 
	\begin{lemma}\label{lem:interpolation}
		Every function $f: \F_\fieldcard^\ell \rightarrow \F_\fieldcard$ is a polynomial function represented by a unique polynomial $P_f(X_0,\dots,X_{\ell-1})$ of degree at most $\fieldcard-1$ in each variable.
		In particular,
		\begin{align*}
			P_f(X_0,\dots,X_{\ell-1}) = \sum_{a_0,\dots,a_{\ell-1} \in \F_\fieldcard} f(a_0,\dots,a_{\ell-1}) \prod_{i=0}^{\ell-1} \left(1 - \princhar(X_i - a_i)\right)\,.
		\end{align*}
	\end{lemma}
%%% Local Variables:
%%% mode: latex
%%% TeX-master: "main"
%%% End:
