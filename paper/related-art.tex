\begin{itemize}
\item \cite{CDSS15}: depth optimized sorting algorithm for HE. Reduce the depth from $\mathcal{O}(l\log^2(N))$ of Batcher network to $\mathcal{O}(\log(N) + \log(l))$ for sorting $N$ $l$-bits integers.

\item \cite{EGNS15}: sorting algorithm for HE. Conclusion: average case in the encrypted domain corresponds to the worst-case in plain domain, better use sorting networks.

\item \cite{CKK15,CKK16} (conference and journal extended version) bit wise comparison using SIMD. Algorithm to compute running products.

\item \cite{KLLW18} equality circuit over non-binary fields. Use the Frobenius automorphism to reduce the depth. 
  
\item \cite{NGEG17}: analysis of bit-wise and digit-wise comparison. Conclusion: bit-wise is more efficient because it has depth $\mathcal{O}(\log(l))$ instead of $\mathcal{O}(l)$ for digit-wise comparisons. Does not use the depth-free Frobenius automorphism\dots

\item \cite{JS19}: bit-wise comparison using SIMD. Conclusion: more efficient than without SIMD. Less interesting than \cite{CKK15} and does not even cite it.
 
\item \cite{LKN19}: modified shell sort. Make shell sort more efficient for HE from $\mathcal{O}(n^2)$ to $\mathcal{O}(n^{3/2}\sqrt{\alpha+\log\log n})$ with failure probability of $2^{-\alpha}$. Complexity worst than for Batcher even-odd merge sort network.

\item \cite{TLWRK20}: Digit-wise comparison using SIMD. Reduce complexity of digit-wise comparison withfrom $\mathcal{O}(t^{d})$ to $\mathcal{O}(t^{r})$ for $r < d$ by decomposing each element in several digits. Compare numbers up to $64$ bits. Depth smaller than $\log(t-1) + \log(d) + 1$ (same algo than us, probably the work to compare with).

\item \cite{AINA:NGEG17}: the univariate circuit is used in the context of sorting. There is no formal description of the circuit properties and complexity. Neither the decomposition method of~\cite{TLWRK20} or the lexicographic circuit is used.

\item \cite{PoPETS:SFR20}: the univariate circuit is used in the context of top-$k$ selection. As above, there is no formal description of the circuit properties and complexity. Neither the decomposition method of~\cite{TLWRK20} or the lexicographic circuit is used. Their minimum function is based on the comparison table from~\cite{CDSS15}, but its multiplicative complexity is quadratic in the length of an input array. In our case, it is $O(n \log n)$.
\end{itemize}


  
%%% Local Variables:
%%% mode: latex
%%% TeX-master: "main"
%%% End:
