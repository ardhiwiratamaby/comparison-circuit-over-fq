\ac{FHE} gives the ability to perform any kind of computations directly over encrypted data. It is therefore a natural candidate for privacy-preserving outsourced storage and computation technics. Since Gentry's breakthrough in 2009, it has received a worldwide attention which has resulted in numerous improvements. As so, \ac{FHE} can now be used in practice in many practical scenarii \cite{KL15,BCIV17,BMMP18} and is currently going through a standardization process \cite{HomomorphicEncryptionSecurityStandard}. The different homomorphic encryption schemes can be classified into three main categories:

\begin{itemize}
\item The schemes encrypting their input bit-wise meaning that each bit of the input is encrypted into a different ciphertext. From there, the operations are carried over each bit separately. Examples of such schemes include FHEW \cite{DM15} and TFHE \cite{CGGI16}. These schemes are the most efficient ones in practice if the metric considered is the \emph{total running time}.
  
\item The second category corresponds to word-wise encryption schemes which allow to perform computations in an \ac{SIMD} fashion \cite{SV14}. These schemes allow, not only to encrypt words of more than one bit, but also to encrypt several of them in different slots such as the operations carried over a single ciphertext will be carried over each slot in a manner similar to \ac{SIMD}. Schemes with this features include BGV \cite{BGV12} and BFV \cite{FV12}. Although operations on these schemes are less efficient than for bit-wise encryption schemes, they become more efficient if one considers the number of slots on which the computation was performed. We refer to this metric as the \emph{amortized cost}.

\item The CKKS scheme \cite{CKKS17}, which allows to perform computations over approximated numbers, forms the third category. It is similar to the second category in the sense that one can pack several numbers in parallel. It is actually the scheme which allows to pack the largest number of elements in a single ciphertext, and as a consequence it usually presents the best amortized cost. Unlike previous schemes it encodes complex, and thus real, numbers natively. However computations are not exact which means that the results are only valid up to a certain precision. 
\end{itemize}

Each category of schemes will be more efficient for a certain application. Thus, when comparing the efficiency of the different homomorphic schemes, one must take into account the considered use case.

It is commonly admitted that schemes of the first category are the most efficient ones for generic applications. Since they operate at the bit level they can compute every logical gate very efficiently. The total runnig time being in this case the sum of the times needed to evaluate each gate of the circuit. As so, in order to optimize the computations for a given application, the only possibility is to reduce the length of the critical computational path and parallelise as much as possible. However, as this becomes more and more difficult as the size of the circuit grows, one has is often reduce to optimize only some parts of the circuit by identifying some patterns \cite{ACS20}.

On the other hand, schemes of the second category operate naturally on $p$-ary arithmetic circuits -- i.e. they are very efficient to evaluate polynomial functions over $\mathbb{F}_p$, for a prime $p$ -- but become much less efficient when considering other kind of computations. One advantage of these schemes is that one can use tools from number theory to evaluate more efficienty specific functions which allow to build relatively efficient $p$-ary circuits. Nonetheless, they remain in general less efficient than first category schemes but can become more efficient when one has to evaluate the same function on multiple inputs in parallel. 

CKKS, similarly to second category schemes, is very efficient when operating on arithmetic circuits. However, unlike other schemes which perform modular artihmetic, it performs computations on complex, and thus real, numbers. Although this is an important  advantage for many use cases which use complex/real numbers, it does not allow to simplify the evaluation of certain functions with number theoretic tools as for the second category. However, as explained above, since it offers the largest number of \ac{SIMD} slots, it usually presents the best amortized cost.

Although \ac{FHE} now offers a relatively efficient alternative for secure computations, some functions remain difficult to evaluate efficiently and this, regardless the considered schemes. Step functions, which are required in many \ac{FHE} applications, form a good example of such functions because of their discontinuous nature. The difficulty to evaluate discontinuous functions comes from the hardness to evaluate a quite common and relatively simple function: the comparison function. Although, comparison is an elementary function which is required in many applications; from the simple \emph{Millionaires problem} of Yao \cite{Yao82}, to advance machine learning tasks such as \emph{genome analysis} as proposed in the iDASH competitions\footnote{http://www.humangenomeprivacy.org/2020/index.html}, it remains difficult to evaluate homomorphically.

\subsection{Contributions}
In this work we study the structure of the circuits corresponding to the comparison function for schemes belonging to the second aforementionned category such as BGV and BFV. For this category of schemes there exists essentially two approaches: either compare the two numbers $x$ and $y$ directly by evaluating a bivariate polynomial in $x$ and $y$, or study the sign of the difference $z=x-y$ by evaluating a univariate polynomial in $z$.

By exploiting the structure of these two polynomials, we show that it is possible to evaluate them more efficiently than what was proposed in the state-of-the-art. The benefit of our approach results in significant performance enhancement for both methods. Using our bivariate circuit we can compare two 64-bits integers with an amortized cost of 21ms which is a gain of $40\%$ when compared to the best previously reported results (See Table \ref{table:comparison_circuit_results}). On the other hand, with our univariate circuit we obtain an amortized cost as small as 11ms which is is better by a factor \todo{??} than what can be achieved with CKKS \todo{update depending on experimental results}. 

We also apply the result of our methods to simple popular computational tasks using comparisons: sorting $N$ numbers and computing minimum/maximum of an array. As a result, for $N=64$ we obtain an amortized cost of 6.5s to sort 8-bit integers and 19.2s for 32-bit integers which is faster than state-of-the-art by a factor 9 and 2.5 respectively (see Table \ref{table:sorting_circuit_results}). We can also obtain the minimum of $N=64$ 8-bit integers in about 35min with an amortized cost of 404ms and in about 7h with an amortized cost of 9.57s for 32-bit integers (see Table \ref{table:minimum_circuit_results}).

\subsection{Related Art}
\label{sec:related-art}
Since comparison is a common function required in many applications, its homomorphic evaluation has been the object of several works.
Since inputs are encrypted, one cannot stop the comparison whenever one meets the first difference between most significant bits. As a consequence homomorphic comparison has a complexity which corresponds to the worst-case complexity in the plain domain. However, the actual efficiency of homomorphic comparison depends on the type of scheme considered.

For bit-wise homomorphic encryption scheme, Chillotti and al. showed that one could compare two $d$-bit integers by evaluating a deterministic weighted automata built made of $5d~\texttt{CMux}$ gates. Using TFHE, evaluating a \texttt{CMux} gate takes around $34$ µs on a classical laptop, meaning that one can compare homomorphically two $d$-bit numbers in around $170d$ µs. Note that these estimations correspond to the fastest (leveled) version of TFHE which does not use bootstrapping. If one wants to use the bootstrapped version then the best method requires to evaluate $7d~\texttt{Mux}$ gates, where each gates takes around $13$ ms to be evaluated, which makes a total of $91d$ ms.

Schemes from the second category can use \ac{SIMD} techniques to batch several plaintexts into a single ciphertext \cite{SV14}. Therefore a natural idea would be to pack the bits of the inputs into a single ciphertext. In \cite{CKK15,CKK16}, Cheon et al. studied comparisons in this context using the bivariate approach. Some of the algorithmic tools they have used -- e.g. computation of running sums and products -- are optimal in the homomorphic setting and have laid the ground for future works in this direction.
Some works have tried to exploit the other feature of these schemes by encoding words modulo a prime $p$ with $p\geq 2$ instead of bits. In \cite{NGEG17}, Narumanchi et al. compare the efficiency of performing comparisons with the univariate approach using integer-wise with bit-wise arithmetic without using batching technique. They concluded that bit-wise methods were more efficient because in this case one can compare $d$-bit numbers with a circuit of depth of $\mathcal{O}(\log d)$ instead of $\mathcal{O}(d)$ in the case of integer arithmetic. This comes from the fact that the integer arithmetic circuit requires to evaluate a Lagrange polynomial of degree $p-1 \geq 2^d$.
In \cite{KLLW18}, Kim et al. noticed that when using batching technique one could take advantage of the nature of $\mathbb{F}_{p^d}$, which corresponds to the plaintext space of each slot, to evaluate the Frobenius automorphism $x \mapsto x^{p}$ without consuming any depth level. This allowed them to reduce the depth of the equality circuit $\texttt{EQ}(x,y) = 1 - (x-y)^{p^{d}-1}$ from $\lceil d\log_2(p) \rceil$ to $\lceil \log_2(d)\rceil + \lceil\log_2(p-1)\rceil$.
Tan et al. proposed a method to perform digit-wise comparison using \ac{SIMD} with the bivariate approach \cite{TLWRK20}. Their idea consists in decomposing the integers to compare into digits of size $p^r$ encoded into a subfield of $\mathbb{F}_{p^d}$, with $r | d$, in order to reduce the degree of the Lagrange polynomial used to interpolate the comparison function. Then one can compute the comparison of the inputs by combining the results of the comparison of each digits using the lexicographical order. Note that their evaluation of the lexicographical order makes intensive use of the efficient equality circuit of \cite{KLLW18}. Overall, they have used their method to compare integers up to 64-bit while reporting, to the best of our knowledge, the current best timings for performing comparisons with BGV scheme. Finally in \cite{PoPETS:SFR20}, Shaul et al. used the univariate approach to evaluate the comparison polynomial in the context of top-$k$ selection with non-binary circuits. However they did not use the decomposition method of \cite{TLWRK20} which leads to relatively poor performance.
Note that all these works did not study the structure of the comparison polynomial neither in the bivariate case nor the univariate case. Unlike aforementioned works, \cite{JMC:KMNN19} studies the polynomial expressions of $\max$, $\argmax$ and other non-arithmetic functions over non-binary fields. However their results do not allow to evaluate these functions very efficiently, as an example their homomorphic circuit to evaluate $\max$ has a quadratic complexity in $p$.

The situation for CKKS is quite different since complex/real numbers are directly encoded into the scheme. Therefore, one does not need to use any finite field encoding but on the other hand one cannot take advantage of the properties resulting of these encoding, such as Frobenius automorphism, neither. Nonetheless the approximated nature of the computations in CKKS makes it suitable to use iterative methods converging to the function one wants to evaluate. In \cite{BMSZ20}, the authors used Newton iteration to evaluate the sign function while independently \cite{AC:CKKLL19,EPRINT:CheKimKim19} studied the efficiency of these methods in a more general context. Using the methods of \cite{EPRINT:CheKimKim19}, one can compare 20-bit numbers with an amortized cost of $5.7$ ms, which is comparable, although slower, to TFHE. However, in order to obtain these timings one has to use already quite large parameters (dimension $2^{17}$ and ciphertext modulus up to $2200$ bits). Since the cost of the operations increases quasi-linearly with the dimension, while the number of slots only increases linearly, increasing the dimension would affect significantly the timings. Therefore it would be interesting to know whether these methods can be used in practice to compare larger inputs -- e.g. 64 bits -- without degrading the performance. 

% \begin{itemize}
% \item \cite{CDSS15}: depth optimized sorting algorithm for HE. Reduce the depth from $\mathcal{O}(l\log^2(N))$ of Batcher network to $\mathcal{O}(\log(N) + \log(l))$ for sorting $N$ $l$-bits integers.

% \item \cite{EGNS15}: sorting algorithm for HE. Conclusion: average case in the encrypted domain corresponds to the worst-case in plain domain, better use sorting networks.

% \item \cite{CKK15,CKK16} (conference and journal extended version) bit wise comparison using SIMD. Algorithm to compute running products.

% \item \cite{KLLW18} equality circuit over non-binary fields. Use the Frobenius automorphism to reduce the depth. 
  
% \item \cite{NGEG17}: analysis of bit-wise and digit-wise comparison. Conclusion: bit-wise is more efficient because it has depth $\mathcal{O}(\log(l))$ instead of $\mathcal{O}(l)$ for digit-wise comparisons. Does not use the depth-free Frobenius automorphism\dots

% \item \cite{JS19}: bit-wise comparison using SIMD. Conclusion: more efficient than without SIMD. Less interesting than \cite{CKK15} and does not even cite it.

% \item \cite{AC:CKKLL19,EPRINT:CheKimKim19}: comparison and min/max functions with CKKS.

% \item \cite{AC:CGGI17}: TFHE-based min/max functions. It is faster than we thought. The fastest algorithm to compare two $n$-bit integers takes $170n$ microseconds, which is comparable to our timings.
 
% \item \cite{LKN19}: modified shell sort. Make shell sort more efficient for HE from $\mathcal{O}(n^2)$ to $\mathcal{O}(n^{3/2}\sqrt{\alpha+\log\log n})$ with failure probability of $2^{-\alpha}$. Complexity worst than for Batcher even-odd merge sort network.

% \item \cite{TLWRK20}: Digit-wise comparison using SIMD. Reduce complexity of digit-wise comparison from $\mathcal{O}(t^{d})$ to $\mathcal{O}(t^{r})$ for $r < d$ by decomposing each element in several digits. Compare numbers up to $64$ bits. Depth smaller than $\log(t-1) + \log(d) + 1$ (same algo than us, probably the work to compare with).

% \item \cite{AINA:NGEG17}: the univariate circuit is used in the context of sorting. There is no formal description of the circuit properties and complexity. Neither the decomposition method of~\cite{TLWRK20} or the lexicographic circuit is used.

% \item \cite{PoPETS:SFR20}: the univariate circuit is used in the context of top-$k$ selection. As above, there is no formal description of the circuit properties and complexity. Neither the decomposition method of~\cite{TLWRK20} or the lexicographic circuit is used. Their minimum function is based on the comparison table from~\cite{CDSS15}, but its multiplicative complexity is quadratic in the length of an input array. In our case, it is $O(n \log n)$.

% \item \cite{JMC:KMNN19}: this work studies polynomial expressions on $\max$, $\argmax$ and other non-arithmetic functions on finite fields. The homomorphic circuit of $\max$ derived from these expressions has a quadratic complexity in $p$. 

% \end{itemize}


  
%%% Local Variables:
%%% mode: latex
%%% TeX-master: "main_pets"
%%% End:





%%% Local Variables:
%%% mode: latex
%%% TeX-master: "main"
%%% End:
