\ac{FHE} gives the ability to perform any kind of computations directly on encrypted data. 
It is therefore a natural candidate for privacy-preserving outsourced storage and computation techniques. 
Since Gentry's breakthrough in 2009~\cite{STOC:Gentry09}, FHE has received a worldwide attention which has resulted in numerous improvements. 
As a result, \ac{FHE} can now be used in practice in many practical scenarios, e.g. genome analysis~\cite{KL15}, energy forecasting~\cite{BCIV17}, image recognition~\cite{BMMP18} and secure messaging~\cite{SP:ACLS18}.\todo{more citations?} 
In addition, FHE is currently going through a standardization process~\cite{HomomorphicEncryptionSecurityStandard}.
  
In practice, homomorphic encryption schemes can be classified into three main categories:
\begin{itemize}
	\item The schemes encrypting their input bit-wise meaning that each bit of the input is encrypted into a different ciphertext. 
	From there, the operations are carried over each bit separately. 
	Examples of such schemes include FHEW \cite{DM15} and TFHE \cite{CGGI16}. 
	These schemes are believed to be the most efficient ones in practice if the metric considered is the \emph{total running time}.
	\item The second category corresponds to word-wise encryption schemes that allow to pack multiple data values into one ciphertext and perform computations on these values in a \ac{SIMD} fashion \cite{SV14}. 
	In particular, encrypted values are packed in different slots such that the operations carried over a single ciphertext are automatically carried over each slot independently. 
	Schemes with these features include BGV~\cite{BGV12} and BFV~\cite{C:Brakerski12,FV12}. 
	Although homomorphic operations in these schemes are less efficient than for bit-wise encryption schemes, their cost per SIMD slot can be better than of the binary-friendly schemes above. 
	We refer to this perfomance metric as the \emph{amortized cost}.
	\item The CKKS scheme~\cite{CKKS17}, which allows to perform computations over approximated numbers, forms the third category. 
	It is similar to the second category in the sense that one can pack several numbers and compute on them in a SIMD fashion.
	The CKKS scheme does not have the algebraic constraints that lower the packing capacity of BGV and BFV. 
	Hence, it is usually possible to pack more elements in a single ciphertext in CKKS, thus resulting the best amortized cost. 
	Unlike previous schemes, CKKS encodes complex, and thus real, numbers natively. 
	However, homomorphic computations are not exact, which means that decrypted results are only valid up to a certain precision. 
\end{itemize}

Each category of schemes is more efficient for a certain application. 
Thus, when comparing the efficiency of different homomorphic schemes, one must take into account a given use case.

It is commonly admitted that schemes of the first category are the most efficient ones for generic applications. 
Since they operate at the bit level, they can compute every logical gate very efficiently. 
The total running time being in this case the sum of the times needed to evaluate each gate of the circuit. 
As a result, to optimize the computations for a given application, the only possibility is to reduce the length of the critical computational path and parallelize the related circuit as much as possible. 
However, as this becomes more and more difficult as the size of the circuit grows, it is possible to optimize only some parts of the circuit by identifying some patterns~\cite{ACS20}.
Another advantage of these schemes is that they have very fast so-called `bootstrapping' algorithms that `refresh' ciphertexts for further computation.
This is very convenient in practice as one can set a standard set of encryption parameters without knowing what function should be computed. 

Schemes of the second category operate naturally on $p$-ary arithmetic circuits, i.e. they are very efficient to evaluate polynomial functions over $\F_p$, for a prime $p$.
However, these schemes become much less efficient when considering other kinds of computations, e.g. comparison operations, step functions. 
To alleviate this problem, one can use tools from number theory to evaluate specific functions with relatively efficient $p$-ary circuits. 
Nonetheless, in general this techniques are too weak to outperform schemes of the first category.
Bootstrapping algorithms of these schemes are quite heavy and usually avoided in practice. 

CKKS, similarly to second category schemes, is very efficient when operating on arithmetic circuits. 
However, unlike other schemes which perform modular arithmetic, it allows to perform computations on complex (and thus real) numbers. 
Although this is an important advantage for many use cases, CKKS lacks simplification tools for evaluation of certain functions due to number-theoretic phenomena as for the second category. 
However, since CKKS usually supports huge packing capacity, it usually presents the best amortized cost.
The bootstrapping algorithm of CKKS is fundamentally different from the above schemes as it refreshes ciphertexts only partially and introduces additional loss of output precision.
Therefore, the CKKS bootstrapping is usually avoided in practice. 

Although \ac{FHE} now offers a relatively efficient alternative for secure computation, some functions remain difficult to evaluate efficiently regardless a scheme. 
Step functions, which are required in many practical applications, form a good example of such functions because of their discontinuous nature. 
The difficulty to evaluate discontinuous functions comes from the hardness to evaluate a quite basic and relatively simple function: the comparison function. 
Although comparison is an elementary operation required in many applications including the famous \emph{Millionaires problem} of Yao \cite{Yao82} or advance machine learning tasks of the iDASH competition\footnote{http://www.humangenomeprivacy.org/2020/index.html}, it remains difficult to evaluate homomorphically.

By now, schemes of the first category look much more suitable for such non-arithmetic tasks, but they are hopelessly inefficient for evaluating arithmetic functions.
Hence, one should resort to heavy conversion algorithms~\cite{JMC:BGGJ20} to leverage the properties of different schemes. 

\subsection{Contributions}
In this work we study the structure of the circuits corresponding to comparison functions for the BGV and BFV schemes. 
For theses schemes, there exists two approaches: either compare two numbers $x$ and $y$ directly by evaluating a bivariate polynomial in $x$ and $y$, or study the sign of the difference $z=x-y$ by evaluating a univariate polynomial in $z$.

By exploiting the structure of these two polynomials, we show that it is possible to evaluate them more efficiently than what was proposed in the state of the art. 
The benefit of our approach results in significant performance enhancement for both methods. 
On the one hand, our bivariate circuit can compare two 64-bit integers with an amortized cost of 21ms, which is a gain of $40\%$ with relation to the best previously reported results of Tan et al.~\cite{TLWRK20} (See Table \ref{table:comparison_circuit_results}). 
On the other hand, our univariate circuit shows even better results with an amortized cost of 11ms for 64-bit numbers -- which is, to the best of our knowledge, more than 3 times faster than previously reported results for this kind of scheme~\cite{TLWRK20}. 
Note that we can compare two 20-bit numbers with an amortized cost of 3ms, which is better by a factor 1.9 than what can be achieved with CKKS-based algorithms and is comparable to TFHE-based implementations (see Table \ref{table:other_he_schemes}).

We also apply our comparison methods to speed up popular computational tasks such as sorting and computing minimum/maximum of an array with $N$ elements. 
For example, for $N=64$, we obtain an amortized cost of 6.5 seconds to sort 8-bit integers and 19.2 seconds for 32-bit integers, which is faster than the prior work by a factor 9 and 2.5 respectively (see Table \ref{table:sorting_circuit_results}). 
For $N=64$, we can also obtain the minimum of 8-bit integers with an amortized running time of 404 ms and of 32-bit integers with an amortized time of 9.57 seconds (see Table \ref{table:minimum_circuit_results}).

\subsection{Related Art}
\label{sec:related-art}
Since comparison is a common function required in many applications, its homomorphic evaluation has been the object of several works.
Since inputs are encrypted, one cannot stop the comparison whenever one meets the first difference between most significant bits. As a consequence homomorphic comparison has a complexity which corresponds to the worst-case complexity in the plain domain. However, the actual efficiency of homomorphic comparison depends on the type of scheme considered.

For bit-wise homomorphic encryption scheme, Chillotti and al. showed that one could compare two $d$-bit integers by evaluating a deterministic weighted automata built made of $5d~\texttt{CMux}$ gates. Using TFHE, evaluating a \texttt{CMux} gate takes around $34$ µs on a classical laptop, meaning that one can compare homomorphically two $d$-bit numbers in around $170d$ µs. Note that these estimations correspond to the fastest (leveled) version of TFHE which does not use bootstrapping. If one wants to use the bootstrapped version then the best method requires to evaluate $7d~\texttt{Mux}$ gates, where each gates takes around $13$ ms to be evaluated, which makes a total of $91d$ ms.

Schemes from the second category can use \ac{SIMD} techniques to batch several plaintexts into a single ciphertext \cite{SV14}. Therefore a natural idea would be to pack the bits of the inputs into a single ciphertext. In \cite{CKK15,CKK16}, Cheon et al. studied comparisons in this context using the bivariate approach. Some of the algorithmic tools they have used -- e.g. computation of running sums and products -- are optimal in the homomorphic setting and have laid the ground for future works in this direction.
Some works have tried to exploit the other feature of these schemes by encoding words modulo a prime $p$ with $p\geq 2$ instead of bits. In \cite{NGEG17}, Narumanchi et al. compare the efficiency of performing comparisons with the univariate approach using integer-wise with bit-wise arithmetic without using batching technique. They concluded that bit-wise methods were more efficient because in this case one can compare $d$-bit numbers with a circuit of depth of $\mathcal{O}(\log d)$ instead of $\mathcal{O}(d)$ in the case of integer arithmetic. This comes from the fact that the integer arithmetic circuit requires to evaluate a Lagrange polynomial of degree $p-1 \geq 2^d$.
In \cite{KLLW18}, Kim et al. noticed that when using batching technique one could take advantage of the nature of $\mathbb{F}_{p^d}$, which corresponds to the plaintext space of each slot, to evaluate the Frobenius automorphism $x \mapsto x^{p}$ without consuming any depth level. This allowed them to reduce the depth of the equality circuit $\texttt{EQ}(x,y) = 1 - (x-y)^{p^{d}-1}$ from $\lceil d\log_2(p) \rceil$ to $\lceil \log_2(d)\rceil + \lceil\log_2(p-1)\rceil$.
Tan et al. proposed a method to perform digit-wise comparison using \ac{SIMD} with the bivariate approach \cite{TLWRK20}. Their idea consists in decomposing the integers to compare into digits of size $p^r$ encoded into a subfield of $\mathbb{F}_{p^d}$, with $r | d$, in order to reduce the degree of the Lagrange polynomial used to interpolate the comparison function. Then one can compute the comparison of the inputs by combining the results of the comparison of each digits using the lexicographical order. Note that their evaluation of the lexicographical order makes intensive use of the efficient equality circuit of \cite{KLLW18}. Overall, they have used their method to compare integers up to 64-bit while reporting, to the best of our knowledge, the current best timings for performing comparisons with BGV scheme. Finally in \cite{PoPETS:SFR20}, Shaul et al. used the univariate approach to evaluate the comparison polynomial in the context of top-$k$ selection with non-binary circuits. However they did not use the decomposition method of \cite{TLWRK20} which leads to relatively poor performance.
Note that all these works did not study the structure of the comparison polynomial neither in the bivariate case nor the univariate case. Unlike aforementioned works, \cite{JMC:KMNN19} studies the polynomial expressions of $\max$, $\argmax$ and other non-arithmetic functions over non-binary fields. However their results do not allow to evaluate these functions very efficiently, as an example their homomorphic circuit to evaluate $\max$ has a quadratic complexity in $p$.

The situation for CKKS is quite different since complex/real numbers are directly encoded into the scheme. Therefore, one does not need to use any finite field encoding but on the other hand one cannot take advantage of the properties resulting of these encoding, such as Frobenius automorphism, neither. Nonetheless the approximated nature of the computations in CKKS makes it suitable to use iterative methods converging to the function one wants to evaluate. In \cite{BMSZ20}, the authors used Newton iteration to evaluate the sign function while independently \cite{AC:CKKLL19,EPRINT:CheKimKim19} studied the efficiency of these methods in a more general context. Using the methods of \cite{EPRINT:CheKimKim19}, one can compare 20-bit numbers with an amortized cost of $5.7$ ms, which is comparable, although slower, to TFHE. However, in order to obtain these timings one has to use already quite large parameters (dimension $2^{17}$ and ciphertext modulus up to $2200$ bits). Since the cost of the operations increases quasi-linearly with the dimension, while the number of slots only increases linearly, increasing the dimension would affect significantly the timings. Therefore it would be interesting to know whether these methods can be used in practice to compare larger inputs -- e.g. 64 bits -- without degrading the performance. 

% \begin{itemize}
% \item \cite{CDSS15}: depth optimized sorting algorithm for HE. Reduce the depth from $\mathcal{O}(l\log^2(N))$ of Batcher network to $\mathcal{O}(\log(N) + \log(l))$ for sorting $N$ $l$-bits integers.

% \item \cite{EGNS15}: sorting algorithm for HE. Conclusion: average case in the encrypted domain corresponds to the worst-case in plain domain, better use sorting networks.

% \item \cite{CKK15,CKK16} (conference and journal extended version) bit wise comparison using SIMD. Algorithm to compute running products.

% \item \cite{KLLW18} equality circuit over non-binary fields. Use the Frobenius automorphism to reduce the depth. 
  
% \item \cite{NGEG17}: analysis of bit-wise and digit-wise comparison. Conclusion: bit-wise is more efficient because it has depth $\mathcal{O}(\log(l))$ instead of $\mathcal{O}(l)$ for digit-wise comparisons. Does not use the depth-free Frobenius automorphism\dots

% \item \cite{JS19}: bit-wise comparison using SIMD. Conclusion: more efficient than without SIMD. Less interesting than \cite{CKK15} and does not even cite it.

% \item \cite{AC:CKKLL19,EPRINT:CheKimKim19}: comparison and min/max functions with CKKS.

% \item \cite{AC:CGGI17}: TFHE-based min/max functions. It is faster than we thought. The fastest algorithm to compare two $n$-bit integers takes $170n$ microseconds, which is comparable to our timings.
 
% \item \cite{LKN19}: modified shell sort. Make shell sort more efficient for HE from $\mathcal{O}(n^2)$ to $\mathcal{O}(n^{3/2}\sqrt{\alpha+\log\log n})$ with failure probability of $2^{-\alpha}$. Complexity worst than for Batcher even-odd merge sort network.

% \item \cite{TLWRK20}: Digit-wise comparison using SIMD. Reduce complexity of digit-wise comparison from $\mathcal{O}(t^{d})$ to $\mathcal{O}(t^{r})$ for $r < d$ by decomposing each element in several digits. Compare numbers up to $64$ bits. Depth smaller than $\log(t-1) + \log(d) + 1$ (same algo than us, probably the work to compare with).

% \item \cite{AINA:NGEG17}: the univariate circuit is used in the context of sorting. There is no formal description of the circuit properties and complexity. Neither the decomposition method of~\cite{TLWRK20} or the lexicographic circuit is used.

% \item \cite{PoPETS:SFR20}: the univariate circuit is used in the context of top-$k$ selection. As above, there is no formal description of the circuit properties and complexity. Neither the decomposition method of~\cite{TLWRK20} or the lexicographic circuit is used. Their minimum function is based on the comparison table from~\cite{CDSS15}, but its multiplicative complexity is quadratic in the length of an input array. In our case, it is $O(n \log n)$.

% \item \cite{JMC:KMNN19}: this work studies polynomial expressions on $\max$, $\argmax$ and other non-arithmetic functions on finite fields. The homomorphic circuit of $\max$ derived from these expressions has a quadratic complexity in $p$. 

% \end{itemize}


  
%%% Local Variables:
%%% mode: latex
%%% TeX-master: "main_pets"
%%% End:





%%% Local Variables:
%%% mode: latex
%%% TeX-master: "main"
%%% End:
