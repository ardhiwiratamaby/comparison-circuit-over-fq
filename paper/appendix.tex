\section{Proof of Theorem \ref{thm:less_than_total_degree}}
\label{proof_thm_less_than_total_degree}
To prove this statement we will need the following Lemma.

  \begin{lemma}\label{lem:difference_to_p-1}
    For all $(a,b)\in\Z^2$ we have:
    \[
      (a - b)^{p-1} = \sum_{i=0}^{p-1} a^i b^{p-1-i} \mod p.
    \]
  \end{lemma}
  \begin{proof}
    Using the binomial theorem we obtain
    \[
      (a - b)^{p-1} = \sum_{i=0}^{p-1} \binom{p-1}{i} a^i (-b)^{p-1-i}.
    \]
    Computing the binomial coefficient modulo $p$
    \begin{align*}
      \binom{p-1}{i} & = \frac{(p-1)!}{i! (p-1-i)!} \\
      & = \frac{(p-1)(p-2)\dots(i+1)}{1 \cdot 2 \dots (p-(1+i))} = (-1)^{p-1-i} \mod p\,,
    \end{align*}
    we prove the lemma.
  \end{proof}

  \begin{lemma}\label{lem:sum_poly}
    Let $P(x)$ be a polynomial of degree $d$ less than $p-1$.
    For any prime number $p > 2$, it holds
    \[
      \sum_{a=0}^{p-1} P(a) = 0 \mod p.
    \]
  \end{lemma}
  \begin{proof}
    It is enough to prove that the sum $\sum_{a=0}^{p-1} a^n = 0 \mod p$ for any $0 \leq n < p-1$. The case $n=0$ is straightforward, now let us assume $n>0$. Let $g$ be a primitive element of $\F_p$.
    Since $p > 2$, we have $g \ne 1$.
    Thus, we can rewrite the above sum as follows.
    \begin{align}
      \sum_{i=1}^{p-1} g^{in} = \frac{g^{pn} - g^n}{g^n - 1}.
    \end{align}
    Since $g^{pn} \equiv g^n \mod p$, the sum turns into zero modulo $p$.
  \end{proof}

  We can now start the proof of Theorem \ref{thm:less_than_total_degree}:
  
  \begin{proof}[Theorem \ref{thm:less_than_total_degree}]
    Assume that all computations are done modulo $p$.
    Using Lemma~\ref{lem:difference_to_p-1}, we obtain
    \begin{align*}
      P_{\LT_\S}(x,y) &=  \sum_{a = 0}^{p-2} \left(1-\sum_{i=0}^{p-1} x^i a^{p-1-i}\right) \sum_{b=a+1}^{p-1} \left(1-\sum_{j=0}^{p-1} y^j b^{p-1-j}\right)\\
    \end{align*}
    Let us expand this expression distributively.
    \begin{align*}
      \sum_{a = 0}^{p-2} \sum_{b=a+1}^{p-1} 1 - \sum_{i=0}^{p-1} x^i a^{p-1-i} - \sum_{i=0}^{p-1} y^i b^{p-1-i} + \sum_{i=0}^{p-1} \sum_{j=0}^{p-1} x^i y^j a^{p-1-i} b^{p-1-j}.
    \end{align*}
    Let us separately compute polynomial coefficients.
    The constant term is equal to
    \begin{align*}
      \sum_{a = 0}^{p-2} \sum_{b=a+1}^{p-1} 1 - a^{p-1} - b^{p-1} + a^{p-1} b^{p-1} = \sum_{a = 0}^{p-2} \sum_{b=a+1}^{p-1} 1 - a^{p-1} - 1 + a^{p-1} = 0 
    \end{align*}
    The coefficients by $x^i$ with $i > 0$ can be computed as follows
    \begin{align*}
      \sum_{a = 0}^{p-2} \sum_{b=a+1}^{p-1} \left(-a^{p-1-i}\right) + a^{p-1-i} = 0.
    \end{align*}
    Next, we compute the coefficients by $y^i$ with $i > 0$.
    \begin{align*}
      -\sum_{a = 0}^{p-2} \sum_{b=a+1}^{p-1} b^{p-1-i} - a^{p-1} b^{p-1-i} &= -\sum_{b=1}^{p-1} b^{p-1-i} - \sum_{a = 1}^{p-2} \sum_{b=a+1}^{p-1} b^{p-1-i} - b^{p-1-i} \\
      &= -\sum_{b=1}^{p-1} b^{p-1-i}.
    \end{align*}
    If $i = p-1$, this sum is equal to $1$.
    According to Lemma~\ref{lem:sum_poly}, it is $0$ if $i \ne 0 \mod (p-1)$.
    
    To compute coefficients by $x^i y^j$ with $i, j > 0$, we will use Faulhaber's formula below
    \begin{align*}
      \sum_{k=1}^n k^e = \frac{1}{e+1} \sum_{i=1}^{e+1} (-1)^{\delta_{ie}} \binom{e+1}{i} B_{e+1-i} \cdot n^i\,,
    \end{align*}
    where $\delta_{ie}$ is the Kronecker delta and $B_{i}$ is the $i$th Bernoulli number.
    This implies that there exist a polynomial $P(x) \in \F_p[X]$ of degree $e+1$ such that
    \begin{align}\label{eq:faulhaber}
      \sum_{k=1}^n k^e = P(n).
    \end{align}
    Note that $P(0) = 0$.
    The coefficient by $x^i y^j$ for some positive $i$ and $j$ is equal to
    \begin{align*}
      \sum_{a = 0}^{p-2} \sum_{b=a+1}^{p-1} a^{p-1-i} b^{p-1-j} = \sum_{b = 1}^{p-1} b^{p-1-j} \sum_{a=0}^{b-1} a^{p-1-i}.
    \end{align*}
    According to~(\ref{eq:faulhaber}), it follows that there exist a polynomial $P(x)$ of degree $p-i$ such that $\sum_{a=0}^{b-1} a^{p-1-i} = P(b)$.
    Since $Q(x) = x^{p-1-j} P(x)$ has degree $2p-1-i-j$, Lemma~\ref{lem:sum_poly} implies that if $i+j > p$, then
    \begin{align*}
      \sum_{b = 1}^{p-1} b^{p-1-j} \sum_{a=0}^{b-1} a^{p-1-i} = \sum_{b=1}^{p-1} Q(b) = 0.
    \end{align*}
    Thus, the total degree of $P_{\LT_\S}$ is at most $p$.

    In addition, we consider the case when $i = j$.
    Let us consider the following sum
    \begin{align*}
      \sum_{a=0}^{p-1} a^{p-1-i} \sum_{b=0}^{p-1} b^{p-1-i} = 0.
    \end{align*}
    We can rewrite it as follows
    \begin{align*}
      \sum_{a=0}^{p-1} a^{p-1-i} \sum_{b=0}^{p-1} b^{p-1-i} = 2\sum_{a=0}^{p-2} a^{p-1-i} \sum_{b=a+1}^{p-1} b^{p-1-i} + \sum_{a=0}^{p-1} a^{2(p-1-i)}.
    \end{align*}
    If $i \ne (p-1)/2$, then the last sum is zero.
    Thus, the coefficient by $x^i y^i$ is equal to $\sum_{a=0}^{p-2} a^{p-1-i} \sum_{b=a+1}^{p-1} b^{p-1-i} = 0$.
    If $i = (p-1)/2$, then following the above argument, this coefficient is equal to $-(p-1)/2$.
  \end{proof}

\section{Decomposition of $f(x,y)$ for $3\leq 3 \leq 7$}
\label{app:decomposition-f}
We remember that we note $Y = y(x-y)$.
\paragraph{p=3}
$$f(x,y) = 2$$

$f$ is constant and can thus be computed without any homomorphic multiplication.

\paragraph{p=5}
$$f(x,y) = 4x^2 + 4x + Y$$

2 homomorphic multiplications to compute $x^2$ and $Y$.

\paragraph{p=7}
$$f(x,y) = 1 + 4x(x+1) + 6[x(x+1)]^2 + (x^2+3x)Y + 6Y^2$$

Which can be computed with only 4 homomorphic multiplications (indicated in bold) when rewritten as:

$$f(x,y) = 1 + 2(\bm{x}^2+x)\bm{\cdot}[2 + 3(x^2+x)] + \bm{Y}\bm{\cdot}[(x^2+3x) + 6Y].$$

\section{Proof of Lemma \ref{lem:univariate}}
\label{app:proof-lem-univariate}
   Let $z = x-y$.
    Thus we can rewrite $Q_{\LT_\S}$ as the univariate indicator function $\chi_{\F_p^-}$, namely
    \begin{align*}
      Q_{\LT_\S}(x,y) = \chi_{\F_p^-}(z) = \sum_{a=-\frac{p-1}{2}}^{-1} 1 - (z - a)^{p-1}.
    \end{align*}
    Thanks to Lemma~\ref{lem:difference_to_p-1}, we can expand $(z-a)^{p-1}$ and obtain
    \begin{align*}
      \sum_{a=-\frac{p-1}{2}}^{-1} 1 - \sum_{i=0}^{p-1} z^i a^{p-1-i}
      = \sum_{i=1}^{p-1} z^i \sum_{a=-\frac{p-1}{2}}^{-1} (-a^{p-1-i}).
    \end{align*}
    If $i$ is even, then the $i$th coefficient is equal to
    \begin{align*}
      -\sum_{a=-\frac{p-1}{2}}^{-1} a^{p-1-i} &= -\sum_{a=1}^{\frac{p-1}{2}} a^{p-1-i} = -\frac{\sum_{a=-\frac{p-1}{2}}^{\frac{p-1}{2}} a^{p-1-i}}{2}.
    \end{align*}
    This coefficient is equal to $0$ for any even $0<i < p-1$ as $\sum_{a=0}^{p-1} a^{p-1-i} = 0$ in this case.
    The $(p-1)$-th coefficient is equal to $-(p-1)/2 = (p+1)/2 \bmod p$.
    If $i$ is odd, then we can rewrite the $i$th coefficient in the following way
    \begin{align*}
      -\sum_{a=-\frac{p-1}{2}}^{-1} a^{p-1-i} &= \sum_{a=1}^{\frac{p-1}{2}} a^{p-1-i},
    \end{align*}
    which finishes the proof.
 

%%%Local Variables:
%%% mode: latex
%%% TeX-master: "main"
%%% End:
