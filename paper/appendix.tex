\section{Proof of Theorem \ref{thm:less_than_total_degree}}
\label{app:proof_thm_less_than_total_degree}
To prove this statement we will need the following Lemma.

  \begin{lemma}\label{lem:difference_to_p-1}
    For all $(a,b)\in\Z^2$ we have:
    \[
      (a - b)^{p-1} = \sum_{i=0}^{p-1} a^i b^{p-1-i} \mod p.
    \]
  \end{lemma}
  \begin{proof}
    Using the binomial theorem we obtain
    \[
      (a - b)^{p-1} = \sum_{i=0}^{p-1} \binom{p-1}{i} a^i (-b)^{p-1-i}.
    \]
    Computing the binomial coefficient modulo $p$
    \begin{align*}
      \binom{p-1}{i} & = \frac{(p-1)!}{i! (p-1-i)!} \\
      & = \frac{(p-1)(p-2)\dots(i+1)}{1 \cdot 2 \dots (p-(1+i))} \\
      &= (-1)^{p-1-i} \mod p\,,
    \end{align*}
    we prove the lemma.
  \end{proof}

  \begin{lemma}\label{lem:sum_poly}
    Let $P(X)$ be a polynomial of degree $d$ less than $p-1$.
    For any prime number $p > 2$, it holds
    \[
      \sum_{a=0}^{p-1} P(a) = 0 \mod p.
    \]
  \end{lemma}
  \begin{proof}
    It is enough to prove that the sum $\sum_{a=0}^{p-1} a^n = 0 \mod p$ for any $0 \leq n < p-1$. The case $n=0$ is straightforward, now let us assume $n>0$. Let $g$ be a primitive element of $\F_p$.
    Since $p > 2$, we have $g \ne 1$.
    Thus, we can rewrite the above sum as follows.
    \begin{align}
      \sum_{i=1}^{p-1} g^{in} = \frac{g^{pn} - g^n}{g^n - 1}.
    \end{align}
    Since $g^{pn} \equiv g^n \mod p$, the sum turns into zero modulo $p$.
  \end{proof}
  Now, We have all the ingredients to prove Theorem \ref{thm:less_than_total_degree}.
  \begin{proof}[Theorem \ref{thm:less_than_total_degree}]
    Assume that all computations are done modulo $p$.
    Using Lemma~\ref{lem:difference_to_p-1}, we obtain that $P_{\LT_\S}(X,Y)$ is equal to
    \begin{align*}
      \sum_{a = 0}^{p-2} \left(1-\sum_{i=0}^{p-1} X^i a^{p-1-i}\right) \sum_{b=a+1}^{p-1} \left(1-\sum_{j=0}^{p-1} Y^j b^{p-1-j}\right)
    \end{align*}
    Let us expand this expression distributively.
    \begin{align*}
      \sum_{a = 0}^{p-2} \sum_{b=a+1}^{p-1} 1 - \sum_{i=0}^{p-1} X^i a^{p-1-i} - \sum_{i=0}^{p-1} Y^i b^{p-1-i} \\
      + \sum_{i=0}^{p-1} \sum_{j=0}^{p-1} X^i Y^j a^{p-1-i} b^{p-1-j}.
    \end{align*}
    Let us separately compute polynomial coefficients.
    The constant term is equal to
    \begin{align*}
      \sum_{a = 0}^{p-2} \sum_{b=a+1}^{p-1} 1 - a^{p-1} - b^{p-1} + a^{p-1} b^{p-1} \\
      = \sum_{a = 0}^{p-2} \sum_{b=a+1}^{p-1} 1 - a^{p-1} - 1 + a^{p-1} = 0 
    \end{align*}
    The coefficients by $X^i$ with $i > 0$ can be computed as follows
    \begin{align*}
      \sum_{a = 0}^{p-2} \sum_{b=a+1}^{p-1} \left(-a^{p-1-i}\right) + a^{p-1-i} = 0.
    \end{align*}
    Next, we compute the coefficients by $Y^i$ with $i > 0$.
    \begin{align*}
      -\sum_{a = 0}^{p-2} &\sum_{b=a+1}^{p-1} b^{p-1-i} - a^{p-1} b^{p-1-i} \\
      &= -\sum_{b=1}^{p-1} b^{p-1-i} - \sum_{a = 1}^{p-2} \sum_{b=a+1}^{p-1} b^{p-1-i} - b^{p-1-i} \\
      &= -\sum_{b=1}^{p-1} b^{p-1-i}.
    \end{align*}
    If $i = p-1$, this sum is equal to $1$.
    According to Lemma~\ref{lem:sum_poly}, it is $0$ if $i \ne 0 \mod (p-1)$.
    
    To compute coefficients by $X^i Y^j$ with $i, j > 0$, we will use Faulhaber's formula below
    \begin{align*}
      \sum_{k=1}^n k^e = \frac{1}{e+1} \sum_{i=1}^{e+1} (-1)^{\delta_{ie}} \binom{e+1}{i} B_{e+1-i} \cdot n^i\,,
    \end{align*}
    where $\delta_{ie}$ is the Kronecker delta and $B_{i}$ is the $i$th Bernoulli number.
    This implies that there exist a polynomial $P(X) \in \F_p[X]$ of degree $e+1$ such that
    \begin{align}\label{eq:faulhaber}
      \sum_{k=1}^n k^e = P(n).
    \end{align}
    Note that $P(0) = 0$.
    The coefficient by $X^i Y^j$ for some positive $i$ and $j$ is equal to
    \begin{align*}
      \sum_{a = 0}^{p-2} \sum_{b=a+1}^{p-1} a^{p-1-i} b^{p-1-j} = \sum_{b = 1}^{p-1} b^{p-1-j} \sum_{a=0}^{b-1} a^{p-1-i}.
    \end{align*}
    According to~(\ref{eq:faulhaber}), it follows that there exist a polynomial $P(X)$ of degree $p-i$ such that $\sum_{a=0}^{b-1} a^{p-1-i} = P(b)$.
    Since $Q(X) = X^{p-1-j} P(X)$ has degree $2p-1-i-j$, Lemma~\ref{lem:sum_poly} implies that if $i+j > p$, then
    \begin{align*}
      \sum_{b = 1}^{p-1} b^{p-1-j} \sum_{a=0}^{b-1} a^{p-1-i} = \sum_{b=1}^{p-1} Q(b) = 0.
    \end{align*}
    Thus, the total degree of $P_{\LT_\S}(X,Y)$ is at most $p$.

    In addition, we consider the case when $i = j$.
    Let us consider the following sum
    \begin{align*}
      \sum_{a=0}^{p-1} a^{p-1-i} \sum_{b=0}^{p-1} b^{p-1-i} = 0.
    \end{align*}
    We can rewrite it as follows
    \begin{align*}
      \sum_{a=0}^{p-1} & a^{p-1-i} \sum_{b=0}^{p-1} b^{p-1-i} \\
      &= 2\sum_{a=0}^{p-2} a^{p-1-i} \sum_{b=a+1}^{p-1} b^{p-1-i} + \sum_{a=0}^{p-1} a^{2(p-1-i)}.
    \end{align*}
    If $i \ne (p-1)/2$, then the last sum is zero.
    Thus, the coefficient by $X^i Y^i$ is equal to $\sum_{a=0}^{p-2} a^{p-1-i} \sum_{b=a+1}^{p-1} b^{p-1-i} = 0$.
    If $i = (p-1)/2$, then following the above argument, this coefficient is equal to $-(p-1)/2$.
  \end{proof}

\section{Decomposition of $f(X,Y)$ for $3\leq p \leq 7$}
\label{app:decomposition-f}
Let $Z = Y(X-Y)$.

\textbf{p=3.}
$$f(X,Y) = 2.$$

Since the polynomial $f(X,Y)$ is constant, it can be computed without any homomorphic multiplication.

\textbf{p=5.}
$$f(X,Y) = 4X^2 + 4X + Z.$$

Two homomorphic multiplications ($\Mul$) are needed to compute $X^2$ and $Z$.

\textbf{p=7.}
$$f(X,Y) = 1 + 4X(X+1) + 6[X(X+1)]^2 + (X^2+3X)Z + 6Z^2$$

In this case, four homomorphic multiplications ($\Mul$) are needed (indicated in bold) when rewritten as follows

$$f(X,Y) = 1 + 2(\bm{X}^2+X)\bm{\cdot}[2 + 3(X^2+X)] + \bm{Z}\bm{\cdot}[(X^2+3X) + 6Z].$$

\section{Proof of Theorem~\ref{th:univariate}}
\label{app:proof-lem-univariate}
   Let $Z = X-Y$.
    Thus we can rewrite $Q_{\LT_\S}(X,Y)$ as the univariate function $\chi_{\F_p^-}$, namely
    \begin{align*}
      Q_{\LT_\S}(X,Y) = \chi_{\F_p^-}(Z) = \sum_{a=-\frac{p-1}{2}}^{-1} 1 - (Z - a)^{p-1}.
    \end{align*}
    Thanks to Lemma~\ref{lem:difference_to_p-1}, we can expand $(Z-a)^{p-1}$ and obtain
    \begin{align*}
      \sum_{a=-\frac{p-1}{2}}^{-1} 1 - \sum_{i=0}^{p-1} Z^i a^{p-1-i}
      = \sum_{i=1}^{p-1} Z^i \sum_{a=-\frac{p-1}{2}}^{-1} (-a^{p-1-i}).
    \end{align*}
    If $i$ is even, then the $i$th coefficient is equal to
    \begin{align*}
      -\sum_{a=-\frac{p-1}{2}}^{-1} a^{p-1-i} &= -\sum_{a=1}^{\frac{p-1}{2}} a^{p-1-i} = -\frac{\sum_{a=-\frac{p-1}{2}}^{\frac{p-1}{2}} a^{p-1-i}}{2}.
    \end{align*}
    This coefficient is equal to $0$ for any even $0<i < p-1$ as $\sum_{a=0}^{p-1} a^{p-1-i} = 0$ in this case.
    The $(p-1)$-th coefficient is equal to $-(p-1)/2 = (p+1)/2 \bmod p$.
    If $i$ is odd, then we can rewrite the $i$th coefficient in the following way
    \begin{align*}
      -\sum_{a=-\frac{p-1}{2}}^{-1} a^{p-1-i} &= \sum_{a=1}^{\frac{p-1}{2}} a^{p-1-i},
    \end{align*}
    which finishes the proof.
 

%%%Local Variables:
%%% mode: latex
%%% TeX-master: "main"
%%% End:
