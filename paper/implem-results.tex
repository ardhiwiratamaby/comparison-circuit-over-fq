We implemented the lexicographic order algorithm described in Section~\ref{sec:background} using the implementation of the BGV scheme~\cite{BGV12} in the HElib library~\cite{HElib}.
For a fair comparison with the prior work, we also implemented the algorithm of Tan et al.~\cite{TLWRK20}.
The code will be publicly available.
For experiments, we used an ordinary commodity laptop equipped with an Intel Dual-Core i5-7267U CPU (running at 3.1 GHz) and 8 GB of RAM without multi-threading.

To compare our algorithms with the state of the art, we ran the lexicographic order algorithm to compute the less-than function on 64-bit integers.
The results of these experiments are presented in Table~\ref{table:comparison_circuit_results}.
In addition, we ran the direct sorting algorithm and the array minimum function from Section~\ref{sec:applications}, see Tables~\ref{table:sorting_circuit_results} and~\ref{table:minimum_circuit_results}.

In the tables presented below, the following notation is used:
\begin{itemize}
  \item $p$: the plaintext modulus;
  \item $q$: the ciphertext modulus;
  \item $m$: the cyclotomic order of the ring $\intring$;
  \item $n$: the degree of the ring $\intring$;
  \item $d$: the dimension of a slot subspace used for digit encoding;
  \item $\ell$: the dimension of digit vectors encoding input integers over $\F_p^d$;
  \item $k$: the number of input integers encoded in one ciphertext.
\end{itemize}

In all the experiments, the encryption parameters of BGV are chosen according to the following strategy.
The plaintext modulus $p$ is a prime number such that SIMD slots are isomorphic to a finite field.
Next, we choose the order $m$ of $\intring$ with large enough $\phi(m)$ to support our homomorphic algorithms.
In addition, $m$ is chosen to be a prime number such that $p$ has a small order modulo $m$.
The former constraint makes sure that the slot permutation group $\mathcal{H}$ is cyclic, which results in a better performance (for more details, see \cite[Appendix C.3]{GHS12}).
The latter constraint maximizes the number of SIMD slots, thus reducing the amortized running time of our algorithms.

We ran our algorithms with $p$ in the interval $[3,659]$.
However, Table~\ref{table:comparison_circuit_results} contains the results with the best amortized running time and the results for small $p$'s, where the bivariate comparison circuits (ours and from~\cite{TLWRK20}) have a comparable running time to the univariate circuit.
For the sorting and the array minimum applications (Tables~\ref{table:sorting_circuit_results} and~\ref{table:minimum_circuit_results}), we showed only the results with $p$ giving the best running time.

\begin{table}[h]
  \centering
  \begin{tabular*}{.9\textwidth}{@{\extracolsep{\fill} } c c c c c c c}
    \toprule
    $(p,m,n)$ & Type & $(d,\ell)$   &  $\log_2 q$    & $k$ & Total time, s & \makecell{Amortized time \\ per slot, ms} \\
    \cmidrule(lr){1-1}\cmidrule(lr){2-7}
    %$(3,34511,34510)$  & U               & $(1,64)$  & $>320$ & $31$   & 5.18   & 167 \\
    %$(3,34511,34510)$  & U               & $(2,32)$  & $>310$ & $63$   & 7.84   & 124 \\
    %$(3,34511,34510)$  & U               & $(4,16)$  & $>310$ & $126$  & 10.11  & 80 \\
    %$(3,34511,34510)$  & U               & $(8,8)$   & $>360$ & $253$  & 16.43  & 65 \\
    $(3,34511,34510)$  & U               & $(16,4)$  & $>450$ & $507$  & 29.16  & 58 \\
    %$(3,34511,34510)$  & \cite{TLWRK20}  & $(1,41)$  & $>320$ & $49$   & 7.10   & 145 \\
    %$(3,34511,34510)$  & \cite{TLWRK20}  & $(2,21)$  & $>320$ & $96$   & 14.80  & 154 \\
    %$(3,34511,34510)$  & \cite{TLWRK20}  & $(4,11)$  & $>320$ & $184$  & 20.60  & 112 \\
    %$(3,34511,34510)$  & \cite{TLWRK20}  & $(5,9)$   & $>320$ & $225$  & 23.67  & 105 \\
    $(3,34511,34510)$  & \cite{TLWRK20}  & $(6,7)$   & $>310$ & $290$  & 27.37  & 94 \\
    %$(3,34511,34510)$  & \cite{TLWRK20}  & $(11,4)$  & $>380$ & $507$  & 48.25  & 95 \\
    %$(3,34511,34510)$  & B               & $(1,41)$  & $>320$ & $49$   & 6.36   & 130 \\
    %$(3,34511,34510)$  & B               & $(2,21)$  & $>310$ & $96$   & 12.33  & 128 \\
    %$(3,34511,34510)$  & B               & $(4,11)$  & $>310$ & $184$  & 16.69  & 91 \\
    %$(3,34511,34510)$  & B               & $(5,9)$   & $>320$ & $225$  & 18.57  & 83 \\
    %$(3,34511,34510)$  & B               & $(6,7)$   & $>310$ & $290$  & 20.51  & 71 \\
    $(3,34511,34510)$  & B               & $(11,4)$  & $>380$ & $507$  & 34.95  & 69 \\
    %$(5,19531,19530)$  & U               & $(1,41)$  & $>330$ & $68$   & 3.04   & 45 \\
    %$(5,19531,19530)$  & U               & $(2,21)$  & $>330$ & $132$  & 4.38   & 33 \\
    %$(5,19531,19530)$  & U               & $(3,14)$  & $>320$ & $199$  & 5.17   & 26 \\
    %$(5,19531,19530)$  & U               & $(4,11)$  & $>320$ & $253$  & 6.11   & 24 \\
    %$(5,19531,19530)$  & U               & $(6,7)$   & $>340$ & $398$  & 8.71   & 22 \\
    $(5,19531,19530)$  & U               & $(7,6)$   & $>350$ & $465$  & 10.09  & 22 \\
    %$(5,19531,19530)$  & \cite{TLWRK20}  & $(1,28)$  & $>290$ & $99$   & 4.68   & 47 \\
    %$(5,19531,19530)$  & \cite{TLWRK20}  & $(4,7)$   & $>295$ & $398$  & 14.33  & 36 \\
    $(5,19531,19530)$  & \cite{TLWRK20}  & $(7,4)$   & $>310$ & $697$  & 25.33  & 36 \\
    %$(5,19531,19530)$  & B               & $(1,28)$  & $>290$ & $99$   & 3.09   & 31 \\
    %$(5,19531,19530)$  & B               & $(2,14)$  & $>290$ & $199$  & 5.72   & 29 \\
    %$(5,19531,19530)$  & B               & $(4,7)$   & $>295$ & $398$  & 8.98   & 23 \\
    $(5,19531,19530)$  & B               & $(7,4)$   & $>300$ & $697$  & 14.73  & 21 \\
    %$(7,14009,14008)$  & U               & $(1,32)$  & $>330$ & $25$   & 1.88   & 75 \\
    %$(7,14009,14008)$  & U               & $(2,16)$  & $>330$ & $51$   & 3.69   & 72 \\
    %$(7,14009,14008)$  & U               & $(4,8)$   & $>310$ & $103$  & 5.21   & 51 \\
    $(7,14009,14008)$  & U               & $(8,4)$   & $>340$ & $206$  & 9.24   & 45 \\
    %$(7,14009,14008)$  & B               & $(1,23)$  & $>330$ & $35$   & 1.88   & 54 \\
    %$(7,14009,14008)$  & B               & $(2,12)$  & $>330$ & $68$   & 3.67   & 54 \\
    %$(7,14009,14008)$  & B               & $(4,6)$   & $>320$ & $137$  & 5.34   & 39 \\
    %$(7,14009,14008)$  & B               & $(6,4)$   & $>310$ & $206$  & 6.91   & 34 \\
    %$(7,14009,14008)$  & B               & $(8,3)$   & $>340$ & $274$  & 9.32   & 34 \\
    %$(7,20197,19116)$  & U               & $(1,32)$  & $>350$ & $66$   & 5.22   & 79 \\
    %$(7,20197,19116)$  & U               & $(2,16)$  & $>350$ & $132$  & 6.81   & 52 \\
    %$(7,20197,19116)$  & U               & $(4,8)$   & $>350$ & $265$  & 9.27   & 35 \\
    $(7,20197,19116)$  & U               & $(8,4)$   & $>380$ & $531$  & 15.70  & 30 \\
    %$(7,20197,19116)$  & B               & $(1,23)$  & $>350$ & $92$   & 6.55   & 71 \\
    %$(7,20197,19116)$  & B               & $(2,12)$  & $>320$ & $177$  & 10.11  & 57 \\
    %$(7,20197,19116)$  & B               & $(3,8)$   & $>320$ & $265$  & 12.33  & 47 \\
    %$(7,20197,19116)$  & B               & $(4,6)$   & $>320$ & $354$  & 15.37  & 43 \\
    $(7,20197,19116)$  & B               & $(6,4)$   & $>340$ & $531$  & 21.45  & 40 \\
    %$(7,20197,19116)$  & B               & $(8,3)$   & $>380$ & $708$  & 29.67  & 42 \\

    %$(11,15797,15796)$ & U               & $(1,25)$  & $>360$ & $57$   & ?   & ? \\
    %$(11,15797,15796)$ & U               & $(2,13)$  & $>360$ & $110$  & ?   & ? \\
    %$(11,15797,15796)$ & U               & $(3,9)$   & $>370$ & $159$  & ?   & ? \\
    %$(11,15797,15796)$ & U               & $(4,7)$   & $>350$ & $205$  & ?   & ? \\
    %$(11,15797,15796)$ & U               & $(5,5)$   & $>360$ & $287$  & ?  & ? \\
    %$(11,15797,15796)$ & U               & $(7,4)$   & $>360$ & $359$  & ?  & ? \\
    %$(11,15797,15796)$ & U               & $(9,3)$   & $>410$ & $478$  & ?  & ? \\

    %$(11,21963,14640)$ & U               & $(1,25)$  & $>360$ & $73$   & 4.59   & 63 \\
    %$(11,21963,14640)$ & U               & $(2,13)$  & $>360$ & $140$  & 6.64   & 47 \\
    %$(11,21963,14640)$ & U               & $(3,9)$   & $>360$ & $203$  & 8.70   & 43 \\
    %$(11,21963,14640)$ & U               & $(4,7)$   & $>350$ & $261$  & 9.67   & 37 \\
    %$(11,21963,14640)$ & U               & $(5,5)$   & $>360$ & $366$  & 12.43  & 34 \\

    %$(67,26881,26880)$ & U               & $(1,13)$  & $>460$ & $413$  & 8.21   & 20 \\
    %$(67,26881,26880)$ & U               & $(2,7)$   & $>460$ & $768$  & 14.57  & 19 \\
    %$(67,26881,26880)$ & U               & $(4,4)$   & $>460$ & $1344$ & 26.63  & 20 \\
    %$(67,26881,26880)$ & U               & $(5,3)$   & $>480$ & $1792$ & 35.37  & 19 \\

    %$(109,24061,24060)$ & U              & $(1,12)$  & $>480$ & $401$  & 8.99   & 22 \\
    %$(109,24061,24060)$ & U              & $(2,6)$   & $>460$ & $802$  & 16.11  & 20 \\
    %$(109,24061,24060)$ & U              & $(3,4)$   & $>440$ & $1203$ & 21.22  & 18 \\
    %$(109,24061,24060)$ & U              & $(4,3)$   & $>440$ & $1604$ & 29.89  & 19 \\

    %$(131,17293,17292)$ & U              & $(1,11)$  & $>465$ & $524$  & 6.57   & 13 \\
    %$(131,17293,17292)$ & U              & $(2,6)$   & $>465$ & $960$  & 11.81  & 12 \\
    $(131,17293,17292)$ & U               & $(3,4)$   & $>430$ & $1441$ & 15.80  & 11 \\

    %$(167,28057,28056)$ & U              & $(1,11)$  & $>520$ & $850$  & 12.38  & 15 \\
    %$(167,28057,28056)$ & U              & $(2,6)$   & $>520$ & $1558$ & 22.36  & 14 \\
    $(167,28057,28056)$ & U              & $(3,4)$   & $>490$ & $2338$ & 30.49  & 13 \\

    %$(173,30103,30102)$ & U              & $(1,10)$  & $>520$ & $1003$ & 13.36  & 13 \\
    $(173,30103,30102)$ & U               & $(2,5)$   & $>520$ & $2006$ & 23.92  & 12 \\
    %$(173,30103,30102)$ & U              & $(3,4)$   & $>490$ & $2508$ & 32.60  & 13 \\

    %$(401,23029,23028)$ & U              & $(1,9)$   & $>570$ & $852$  & 14.50  & 17 \\
    %$(401,23029,23028)$ & U              & $(3,3)$   & $>540$ & $2558$ & 35.15  & 14 \\
    \bottomrule
  \end{tabular*}
  \caption{The running time of our lexicographic order algorithms and the algorithm of Tan et al.~\cite{TLWRK20} to compare 64-bit integers. The second column (Type) indicates which comparison circuit is used: univariate (U), bivariate (B) or bivaraite from~\cite{TLWRK20}. The total time is averaged over 10 trials.\todo{do more trials}\todo{Security level?}}
  \label{table:comparison_circuit_results}
\end{table}

\begin{table}[h]
  \centering
  \begin{tabular*}{.9\textwidth}{@{\extracolsep{\fill} } c c c c c c}
    \toprule
    $N$     & $\log_2 q$    & \#Trials  & \makecell{Avg. total \\ time, s}    & \makecell{Amortized time \\ per slot, ms} & \makecell{Amortized time \\ per slot, ms \cite{CDSS15}} \\
    \midrule
    \multicolumn{6}{l}{8-bit integers ($d=2$, $\ell=1$, $k=9352$)} \\
    \cmidrule(lr){1-6}
    4       & 589     & 32        & 186.28       & 20    & 140 \\
    8       & 599     & 20        & 867.46       & 93    & 690 \\
    16      & 599     & 20        & 3652.23      & 391   & 3140\\
    32      & 599     & 20        & 14769.23     & 1579  & 13900 \\
    64      & 604     & 10        & 60351.02     & 6453  & 60000 \\
    \midrule
    \multicolumn{6}{l}{32-bit integers ($d=3$, $\ell=2$, $k=4676$)} \\
    \cmidrule(lr){1-6}
    4       & 659     & 20        & 299.17       & 64    & 200 \\
    8       & 671     & 20        & 1356.19      & 290   & 944 \\
    16      & 671     & 20        & 5700.12      & 1219  & 4280 \\
    32      & 684     & 20        & 23017.03     & 4922  & 18600 \\
    64      & 684     & 10        & 89972.27     & 19241 & 49700 \\
    \bottomrule
  \end{tabular*}
  \caption{The running time needed to sort $N$ 8-bit or 32-bit integers with $p=167$, $m=28057$ and $n=28056$. The minimal security level is 92 bits according to the LWE estimator~\cite{lwe_estimator}. Note that the amortized timing per slot from~\cite{CDSS15} is obtained with the LTV scheme~\cite{STOC:LopTroVai12}, which was attacked by Albrecht et al.~\cite{C:AlbBaiDuc16}.}
  \label{table:sorting_circuit_results}
\end{table}

\begin{table}[h]
  \centering
  \begin{tabular*}{.9\textwidth}{@{\extracolsep{\fill} } c c c c c c}
    \toprule
    $N$     & $T$   & $\log_2 q$    & \#Trials  & \makecell{Avg. total \\ time, s}    & \makecell{Amortized time \\ per slot, ms} \\
    \midrule
    \multicolumn{5}{l}{8-bit integers ($d=3$, $\ell=1$, $k=5220$)} \\
    \cmidrule(lr){1-6}
    2       & 1     & 232           & 20        & 12.35     & 2.37 \\
    4       & 2     & 406           & 20        & 50.55     & 9.68 \\
    8       & 3     & 579           & 20        & 151.75    & 29.07 \\
    16      & 4     & 766           & 20        & 386.87    & 74.11 \\
    32      & 3     & 825           & 20        & 883.68    & 169.29 \\
    64      & 3     & 854           & 20        & 2111.56   & 404.51 \\
    \midrule
    \multicolumn{5}{l}{32-bit integers ($d=6$, $\ell=2$, $k=2610$)} \\
    \cmidrule(lr){1-6}
    2       & 1     & 406           & 20        & 37.71     & 14.54 \\
    4       & 2     & 753           & 20        & 157.80    & 60.46 \\
    8       & 1     & 825           & 20        & 506.24    & 193.96 \\
    16      & 1     & 839           & 20        & 1694.15   & 649.10 \\
    32      & 1     & 854           & 20        & 6440.27   & 2467.54 \\
    64      & 1     & 884           & 19        & 24984.11  & 9572.46 \\
    \bottomrule
  \end{tabular*}
  \caption{The running time needed to find the minimum of $N$ 8-bit or 32-bit integers with $p=17$, $m=41761$ and $n=41760$. The parameter $T$ is equal to the number of the tournament method stages. The minimal security level is 121 bits according to the LWE estimator~\cite{lwe_estimator}.}
  \label{table:minimum_circuit_results}
\end{table}

\subsection{Comparison to other HE schemes}
    As mentioned in the introduction, there are three types of HE schemes suitable in different use cases including TFHE, CKKS and BGV/BFV.
    Our algorithms are designed for BGV/BFV, which support SIMD packing and are the most efficient FHE schemes for exact computation in arithmetic circuits.
    However, the amortized running time per data value of our comparison algorithms is comparable to efficient FHE schemes for binary circuits (TFHE) and HE supporting approximate arithmetic over real numbers (CKKS). 

    In Table~\ref{table:other_he_schemes}, our implementation of the less-than function is compared to the implementations of this function in TFHE \cite{AC:CGGI17} and CKKS \cite{EPRINT:CheKimKim19}.

    Chilloti et al. \cite{AC:CGGI17} constructed a deterministic weighted automata that can compute the maximum function using the TFHE scheme.
    The same automata can compute the less-than function without any performance loss. 
    In this case, the running time of the less-than function of two $n_b$-bit integers takes $170 n_b$ microseconds on a hardware similar to ours and with the encryption parameters supporting at least 110 bits of security.\todo{check with relation to the result of Daniele}
    Note that this running time is achieved in the leveled mode of TFHE, i.e. without bootstrapping.

    Cheon et al. \cite{EPRINT:CheKimKim19} designed a polynomial approximation of the less-than function over real numbers.
    Since the precision of this approximation depends on the ciphertext noise which cannot be reduced in CKKS, only a few consecutive comparisons are possible to perform correctly unlike in TFHE and BGV/BFV.
    Nevertheless, the number of SIMD slots in CKKS is always $n/2$, which significantly reduces the amortized running time per value.
    Since the implementation in \cite{EPRINT:CheKimKim19} exploits CPU parallelization with 8 threads and ours avoid multi-threading, we put an estimated running time of \cite{EPRINT:CheKimKim19} for one core in Table~\ref{table:other_he_schemes}.
    The encryption parameters of the CKKS scheme are set to a security level of 128 bits.

    Our implementation runs with $p=131$, $m=17293$ ($\phi(m)=17292$), which corresponds to 5764 integers of 8-16 bits encoded into one ciphertext.
    The encryption parameters corresponds to a security level of 134 bits.
    
    \todo{Be more concrete in the below paragraph once you finish experiments}
    As shown in Table~\ref{table:other_he_schemes}, our algorithm for the less-than function demonstrates a similar performance as the TFHE-based work and outperforms the CKKS-based algorithms.
    This means that in use cases that involve arithmetic and non-arithmetic functions (e.g. artificial neural networks) one might need only an HE scheme supporting arithmetic circuits (i.e. BGV/BFV) instead of combining it with an HE scheme efficient for a non-arithmetic part of the computation (i.e. TFHE).

    \begin{table}[h]
      \centering
      \begin{tabular*}{.9\textwidth}{@{\extracolsep{\fill} } c c c c}
        \toprule
        Bit length  & FHE scheme  & \makecell{Total \\ time, s}    & \makecell{Amortized time \\ per slot, ms} \\
        \midrule
        \multirow{3}{*}{8}  & TFHE              & 0.001     & 1.36 \\
                            & CKKS              & ?31x8     & ?0.47x8 \\
                            & BGV(this paper)   & 7.09      & 1.23 \\
        \midrule
        \multirow{3}{*}{10}  & TFHE             & 0.002     & 1.70 \\
                             & CKKS             & ?47x8     & ?0.72x8 \\
                             & BGV(this paper)  & 7.09      & 1.23 \\
        \midrule
        \multirow{3}{*}{12}  & TFHE             & 0.002     & 2.04 \\
                             & CKKS             & ?80x8     & ?1.22x8 \\
                             & BGV(this paper)  & 7.09      & 1.23 \\
        \midrule
        \multirow{3}{*}{16}  & TFHE             & 0.003     & 2.72 \\
                             & CKKS             & ?94x8     & ?1.43x8 \\
                             & BGV(this paper)  & 12.11     & 2.10 \\ 
        \bottomrule
      \end{tabular*}
      \caption{Total and amortized running time of the maximum function implemented with different HE schemes including TFHE, CKKS and BGV. Note that TFHE does not support SIMD packing, which implies that the total and amortized running time of this scheme are the same.}
      \label{table:other_he_schemes}
    \end{table}
    
%%% Local Variables:
%%% mode: latex
%%% TeX-master: "main"
%%% End:
