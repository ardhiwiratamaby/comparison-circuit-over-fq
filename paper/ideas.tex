The homomorphic circuit for $\LT_{\F_\fieldcard}(x,y)$ is presented in Algorithm~\ref{alg:basic_less_than_circuit}.
Unlike the prior works (\todo{cite}) that evaluate this circuit as a multivariate polynomial, we exploit its representation in~(\ref{eq:less_than_function}) to parallelize computations in the SIMD manner. 
\begin{algorithm}[t]
  \KwIn{
  $\ct_x$ -- a ciphertext encrypting $x \in \F_\fieldcard$ in the first SIMD slot and $0$ in other slots,
  $\ct_y$ -- a ciphertext encrypting $y \in \F_\fieldcard$ in the first SIMD slot and $0$ in other slots,
  $\S = \{a_0,a_1,\dots,a_{\fieldcard-1}\}$ -- a complete set of representatives of $\F_\fieldcard$.
  }
  \KwOut{$\ct$ -- a ciphertext encrypting $1$ in the first SIMD slot if $x < y$, otherwise $0$. Other slots contain $0$.}
  $\ct_1 \leftarrow \Replicate(\ct_x, \fieldcard - 1)$//\todo{Define replicate}\\
  $\ct_2 \leftarrow \Replicate(\ct_y, \fieldcard - 1)$\\
  $\pt_1 \leftarrow \pt(a_0, a_1, \dots, a_{\fieldcard-2})$//\todo{define this operation}\\
  $\pt_2 \leftarrow \pt(a_1, a_2, \dots, a_{\fieldcard-1})$\\
  $\ct_1 \leftarrow \ct_1 - \pt_1$//\todo{define operations between ciphertexts and plaintexts}\\
  $\ct_2 \leftarrow \ct_2 - \pt_2$\\
  $\ct_1 \leftarrow \ct_1^{\fieldcard-1}$//$\princhar_\fieldcard(x-a)$, \todo{define this operation}\\
  $\ct_2 \leftarrow \ct_2^{\fieldcard-1}$//$\princhar_\fieldcard(y-b)$\\
  $\ct_1 \leftarrow 1 - \ct_1$// define operations between constants and ciphertexts\\
  $\ct_2 \leftarrow 1 - \ct_2$\\
  $k \leftarrow 1$// compute all running sums $\sum_{b \in \S, a < b} (1-\princhar_\fieldcard(x - b))$ \\
  \While{$k < \fieldcard-1$}{
    $\ct_{tmp} \leftarrow \Shift(\ct_2, k)$//\todo{define shift}\\
    $\ct_2 \leftarrow \ct_2 + \ct_{tmp}$\\
    $k \leftarrow 2k$\\
  }
  $\ct \leftarrow \ct_1 \cdot \ct_2$\\
  $k \leftarrow 1$\\
  \While{$k < \fieldcard-1$}{
    $\ct_{tmp} \leftarrow \Shift(\ct, k)$\\
    $\ct \leftarrow \ct + \ct_{tmp}$\\
    $k \leftarrow 2k$\\
  }
  $\ct \leftarrow \Select(\ct, \{1\})$//\todo{define select}\\
  \textbf{Return} $\ct$.
  \caption{Homomorphic circuit of $\LT_{\F_\fieldcard}$.}\label{alg:basic_less_than_circuit}
\end{algorithm}
\todo{Make a separate algorithm for running sums (while circuit above)}
\todo{Write down the complexity of Algorithm~\ref{alg:basic_less_than_circuit}, compare with prior works}

Here are potential cyclotomic polynomials to use for a failure probability smaller $\varepsilon < 2^{-32}$.

\begin{table}
  \setlength{\tabcolsep}{1em}
  \centering
  \begin{tabular}{||ccc||cccccc||}
    \hline
    $~m~$ & $~n~$ & $~\delta_\mathcal{R}~$ & $~p~$ & $~d~$ & depth & $~\ell~$ & $~s~$ & $~\varepsilon~$ \\
    \hline
    $22,383$ & $14,904$ & $5n$ & $5$  & $18$ & $8$ & $828$  & $1$ &  $2^{-41.7}$ \\
    $25,623$ & $15,552$ & $118n$ & $2$ & $36$ & $7$ & $432$ & $2$ & $2^{-36.0}$ \\
    $31,132$ & $15,120$ & $85n$ & $7$ & $12$ & $8$ & $1,260$ & $2$ & $2^{-33.6}$ \\
    $31,726$ & $15,288$ & $57n$ & $3$ & $28$ & $7$ & $546$ & $1$ & $2^{-44.3}$ \\
    $35,552$ & $16,000$ & $21n$ & $17$ & $10$ & $9$ & $1,600$ & $3$ & $2^{-40.8}$ \\
    \hline
    $31,697$ & $30,576$ & $57n$ & $3$ & $28$ & $7$ & $1,092$ & $1$ & $2^{-44.3}$ \\
    $48,771$ & $32,508$ & $5n$ & $2$ & $42$ & $7$ & $774$ & $1$ & $2^{-42.0}$ \\
    $57,368$ & $28,000$ & $141n$ & $17$ & $10$ & $9$ & $2,800$ & $3$ & $2^{-40.8}$ \\
    $72,400$ & $28,800$ & $9n$ & $7$ & $12$ & $8$ & $2,400$ & $3$ & $2^{-33.6}$ \\
    $93,006$ & $30,996$ & $5n$ & $5$ & $18$ & $8$ & $1,722$ & $1$ & $2^{-41.7}$ \\
    \hline
                                                   
  \end{tabular}
  \caption{Potential parameters}
  \label{tab:params}
\end{table}

\begin{itemize}
\item $m$: cyclotomic order;
\item $n$: degree of the cyclotomic;
\item $\delta_\mathcal{R}$: expansion factor (upper-bound);
\item $p$: plaintext modulus;
\item $d$: degree of the factors of $\phi_m$ modulo $p$;
\item depth: $\lceil \log_2(p-1) \rceil + \lceil \log_2(d)\rceil + 1$;
\item $\ell$: number of factors of of $\phi_m$ modulo $p$;
\item $s$: number of generators for $(\Z/m\Z)^\times/<p>$;
\item $\varepsilon$: approximative probability of failure for one equality check.
\end{itemize}