%%%%%%%
Let $\S$ be a totally ordered set with a binary relation $<$.
For any $x,y \in \S$, we can define the less-than and the equality functions as follows.
\begin{align*}
  \LT_\S(x,y) = 
  \begin{cases}
    1, & \text{if } x < y; \\
    0, & \text{if } x \ge y, \\
  \end{cases}
  \qquad
  \EQ_\S(x,y) = 
  \begin{cases}
    1, & \text{if } x = y; \\
    0, & \text{if } x \neq y. \\
  \end{cases}
\end{align*}
If $\S$ is a complete set of representatives of a finite field $\F_\fieldcard$, then the equality function can be easily realized using the principal character of $\F_\fieldcard$, namely $\EQ_\S(x,y) = 1 - \princhar(x-y)$.
Since this function is independent of the choice of $\S$, we can write
\begin{align*}
  \EQ_{\F_{\fieldcard}}(x,y) = 1 - \princhar(x-y).
\end{align*}
The less-than function can be interpolated using Lemma~\ref{lem:interpolation} and the following truth table
\begin{align*}
  \begin{array}{c|ccccc}
    < & a_0 & a_1 & a_2 & \cdots & a_{q-1} \\
    \hline
    a_0 & 0 & 1 & 1 & \cdots & 1 \\
    a_1 & 0 & 0 & 1 & \cdots & 1 \\
    a_2 & 0 & 0 & 0 & \cdots & 1 \\
    \vdots & \vdots & \vdots & \vdots & \ddots & \vdots \\
    a_{q-1} & 0 & 0 & 0 & \cdots & 0 \\
  \end{array}
\end{align*}
where $a_i \in \S$ for any $i \in [0,q-1]$ and $a_i < a_j$ if $i < j$.
In particular,
\begin{align*}
  \LT_\S(x,y) &= \sum_{a,b \in \S} \LT_\S(a,b) (1-\princhar(x - a)) (1-\princhar(x - b)) \\
  &= \sum_{a \in \S} (1-\princhar(x - a)) \sum_{b \in \S} \LT_\S(a,b)  (1-\princhar(x - b)) \\
  &= \sum_{a \in \S} (1-\princhar(x - a)) \sum_{b \in \S, a < b} (1-\princhar(x - b)).
\end{align*}
In contrast to equality, this function depends on the choice of $\S$.
Nevertheless, we can fix a certain $\S$ as the default set of representatives and denote $\LT_\S(x,y)$ as $\LT_{\F_\fieldcard}(x,y)$.
\todo{Continue on the less-than function}
%%%%%%%

\todo{Rewrite the following.}
We want to compare two vectors $X=(x_1,x_2,\ldots,x_\ell)$ and $Y=(y_1,y_2,\ldots,y_\ell)\in\mathbb{F}_{p^d}^\ell$ for the lexicographical order $<_L$:

$$X <_L Y \Leftrightarrow \exists i\in[1,\ell] \text{ such that } x_i < y_i \text{ and } \forall j\leq i ~~ x_j = y_j $$

where $x<y$ compare $x$ and $y$ as if they were elements of $[0,p^d-1]$. Let us assume that we can compute the logical gates $<$ and $=$ over $\mathbb{F}_{p^d}$, then we can evaluate the lexicographical order $<_L$ as: 

$$ X <_L Y = \sum_{i=1}^\ell \left( (x_i < y_i).\prod_{j=1}^{i-1} (x_i = y_j) \right)$$

We can evaluate the term $\prod_{j=1}^{i-1} (x_i = y_j)$ using the randomized vector equality testing with a depth independent of $i$. 

Suppose we have two ciphertexts encrypting the vectors of length $q-1$ (can be of length $k\geq q-1$ with $0$s padding) $(x,0,\ldots,0)$ and $(y,0,\ldots,0)$ in an SIMD fashion. 
We can obtain ciphertexts encrypting $(x,x,\ldots, x)$ and $(y,y,\ldots, y)$ with $2\log_2(q-1)\texttt{Rot}$ and $2\log_2(q-1)\texttt{Add}$. 

From there we can obtain encryptions of $(x,x-1,\ldots, x-q-2)$ and $(y-1,y-2,\ldots, y-q-1)$ with $2\texttt{Add}$. We can apply $f$ to these vectors in parallel with a depth of $\log (p-1) + \log d$ for a cost of $2(\log (p-1) + wt(p-1) + d - 2) \texttt{Mult}$. This can be minimized by choosing $p = 2^d + 1$ for some $d$.

With $2$ more $\texttt{Sub}$ we obtain encryptions of $(1-f(x), 1-f(x-1), \ldots, 1-f(x-q-2))$ and $(1-f(y-1), 1-f(y-2), \ldots, 1-f(y-q-1))$. From there we can get an encryption of the different partial sums in $\log_2(q-1)\texttt{Rot}$, $\log_2(q-1)\texttt{Select}$ and $\log_2(q-1)\texttt{Add}$. The algorithm works as follow:

\begin{center}
  \begin{tikzpicture}
    \newcommand\y{-4};
    
    \draw (0,2) -- (16,2);
    \draw (0,0) -- (16,0);    
    \foreach \x in {0,...,8}{
      \draw (2*\x,0) -- (2*\x,2);
    }
    \foreach \x in {0,...,7}{
      \node at (2*\x+1,1) {$x_\x$};
    }

    \draw (0,2+\y) -- (16,2+\y);
    \draw (0,\y) -- (16,\y);    
    \foreach \x in {0,...,8}{
      \draw (2*\x,0+\y) -- (2*\x,2+\y);
    }

    \node at (1,1+\y) {$x_0 + x_1$};
    \node at (3,1+\y) {$x_1 + x_2$};
    \node at (5,1+\y) {$x_2 + x_3$};
    \node at (7,1+\y) {$x_3 + x_4$};
    \node at (9,1+\y) {$x_4 + x_5$};
    \node at (11,1+\y) {$x_5 + x_6$};
    \node at (13,1+\y) {$x_6 + x_7$};
    \node at (15,1+\y) {$x_7$};

    %%%%%%%%%%%%%%%%%%%%%%%%%%%%%%%%%%%%
    
    \draw (0,2+2*\y) -- (16,2+2*\y);
    \draw (0,0+2*\y) -- (16,0+2*\y);    
    \foreach \x in {0,...,8}{
      \draw (2*\x,0+2*\y) -- (2*\x,2+2*\y);
    }

    \node at (1,1+2*\y) {$\begin{array}{c}
        x_0 + x_1  \\
        + x_2 + x_3
      \end{array}$};
    \node at (3,1+2*\y) {$\begin{array}{c}
        x_1 + x_2  \\
        + x_3 + x_4
      \end{array}$};
    \node at (5,1+2*\y) {$\begin{array}{c}
        x_2 + x_3  \\
        + x_4 + x_5
      \end{array}$};
    \node at (7,1+2*\y) {$\begin{array}{c}
        x_3 + x_4  \\
        + x_5 + x_6
      \end{array}$};
    \node at (9,1+2*\y) {$\begin{array}{c}
        x_4 + x_5  \\
        +x_6 + x_7
      \end{array}$};
    \node at (11,1+2*\y) {$\begin{array}{c}
        x_5 + x_6  \\
        + x_7
      \end{array}$};
    \node at (13,1+2*\y) {$x_6 + x_7$};
    \node at (15,1+2*\y) {$x_7$};

    %%%%%%%%%%%%%%%%%%%%%%%%%%%%%%%%%%%%%%%%%%
    
    \draw (0,2+3*\y) -- (16,2+3*\y);
    \draw (0,0+3*\y) -- (16,0+3*\y);    
    \foreach \x in {0,...,8}{
      \draw (2*\x,0+3*\y) -- (2*\x,2+3*\y);
    }

    \node at (1,1+3*\y) {$\begin{array}{c}
        x_0 + x_1  \\
                            + x_2 + x_3 \\
                            + x_4 + x_5 \\
                            + x_6 + x_7
      \end{array}$};
    \node at (3,1+3*\y) {$\begin{array}{c}
        x_1 + x_2  \\
                            + x_3 + x_4 \\
                            + x_5 + x_6 \\
                            + x_7 \\
      \end{array}$};
    \node at (5,1+3*\y) {$\begin{array}{c}
        x_2 + x_3  \\
                            + x_4 + x_5 \\
                            + x_6 + x_7
      \end{array}$};
    \node at (7,1+3*\y) {$\begin{array}{c}
        x_3 + x_4  \\
                            + x_5 + x_6 \\
                            +x_7
      \end{array}$};
    \node at (9,1+3*\y) {$\begin{array}{c}
        x_4 + x_5  \\
        +x_6 + x_7
      \end{array}$};
    \node at (11,1+3*\y) {$\begin{array}{c}
        x_5 + x_6  \\
        + x_7
      \end{array}$};
    \node at (13,1+3*\y) {$x_6 + x_7$};
    \node at (15,1+3*\y) {$x_7$};

    \foreach \x in {2,...,8}
    \draw[->,thick = 2pt] (2*\x-0.1-1,2) to[bend right] (2*\x-3+0.1,2);

    \foreach \x in {3,...,8}
    \draw[->,thick = 2pt] (2*\x-0.1-1,2+\y) to[bend right] (2*\x-5+0.1,2+\y);

    \foreach \x in {5,...,8}
    \draw[->,thick = 2pt] (2*\x-0.1-1,2+2*\y) to[bend right] (2*\x-9+0.1,2+2*\y);

\end{tikzpicture}
\end{center}

From there we just have to compute the scalar product of the two vectors with $1\texttt{Mult} + \log_2(q-1)\texttt{Rot} + \log_2(q-1)\texttt{Add}$. \newline

So overall we can compute $<$ over $\mathbb{F}_q$ for $q = p^d$ in:

\begin{itemize}
\item $4\log_2 (q-1) \texttt{Rot}$
\item $(4\log_2 (q-1) + 4) \texttt{Add}$
\item $(2(\log (p-1) + wt(p-1) + d - 2) + 1) \texttt{Mult}$
\item $(\log_2 (q-1) + 1) \texttt{Select}$
\end{itemize}


%%% Local Variables:
%%% mode: latex
%%% TeX-master: "main"
%%% End:
